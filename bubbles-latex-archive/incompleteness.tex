\chapter{Incompleteness}
Our goal of this section we will prove Turing's Halting Problem and Godel's First and second incompleteness theorem, using 2 assumptions
\begin{enumerate}
    \item Any mathematical system capable of arithmetic on the natural numbers can simulate a general purpose computer, notably both Turing Machines and lambda calculus can be defined using relatively simple operations on functions between the natural numbers.
    \item The correctness of any proof can be verified by a computer in a finite amount of time. This is relatively easy to see since every system of mathematics can be reduced to.
    \begin{itemize}
        \item A finite amount of primative objects. (In the case of ZFC, sets and logical connectives like: $\forall, \wedge, \neg, \implies$)
        \item A finite amount of gramatical methods for combining previous correct statements made of primitive objects to produce new correct statements.
        \item A finite list of axioms that guarentee the truth of a list of statements.
    \end{itemize}
    Under this framework, a computer can just take any proof (that must be finite), and for every symbol in the proof check to make sure that the gramatical methods are correct, and then assert that any axioms used were true. Since this step is finite, and the proof is finite, the method must only take a finite amount of time.
\end{enumerate}
Notably this last component implies that since every single proof is a list of symbols of finite length, then the set of all mathematical proofs is countable.

We begin with turing's halting problem.

For the set $Program$ of all computer programs and $Data$ of all data that could be fed into programs then there is a function $Halt$ described like so:
\begin{align}
    Program \times Data \xrightarrow{Halt} \{0,1\}
\end{align}
where it returns $0$ if the program halts, and $1$ if the program loops
\begin{theorem}[Turing's Halting Problem]
    Halt(P,D) is not computable
\end{theorem}
\begin{proof}
    Let us assume the halting problem does exist as a computable function, then we can define a function $G(P)$ that will loop if $P(P)$ loops and will not loop if $P(P)$ does not loop. Luckily you can write a simple function
\begin{lstlisting}[language=Python]
def G(P):
    return P(P)
\end{lstlisting}
Now our goal is to define $G^*(P)$ that is defined as the "opposite" of $G(P)$ where it halts whenever the program loops, and loops whenever the program halts
\begin{lstlisting}[language=Python]
def G*(P):
    if Halt(P,P)==0:
        while True:
            print("infinite loop")
    else:
        return True
\end{lstlisting}
Using this we know that $G(P)$ halts if and only if $G^*(P)$ does not halt, however we have a contradiction since by definition $G^*(G^*)=G(G^*)$. However they cannot be equal since they have opposite halting behaviors for every program, therefore we get that the function $Halt$ used to construct $G^*$ must not exist.
\end{proof}
\begin{theorem}[Godel's Incompleteness Theorems]
    For any mathematically consistent system with axioms $\Gamma$ that is powerful enough to include arithmetic on finite integers then:
    \begin{enumerate}
        \item There exists a proposition $P$, such that there is no proof of $\Gamma \implies P$ or $\Gamma \implies \neg P$.
        \item The proposition "$\Gamma$ is consistent" (aka, there is no arrow $\Gamma \implies \textbf{False}$) is one such proposition.
    \end{enumerate}
\end{theorem}

    First reminder from above that 
    \begin{enumerate}
        \item The set of all proofs are countable
        \item A computer can quickly check if a proof $P$ is a valid proof of a proposition $T$
        \item Arithmetic can model the inner workings of a computer.
    \end{enumerate}
First we should consider a quick lemma:
\begin{lemma}
    If a $\Gamma$ system of mathematics contains arithmetic. For every program $P$ that halts, there is a proof showing that $P$ halts.
\end{lemma}
\begin{proof}
    Since $$
\end{proof}

Now consider a computer program $S^*$ defined in a similar manner to $G*$ that takes in a computer program $P$ and does the following:
\begin{lstlisting}[language=Python]
def S*(P):
    for each x in Proofs:
        if x is a proof that P(P) halts:
            while True:
                print("infinite loop")
        if x is a proof that P(P) does not halt:
            return True
\end{lstlisting}
\begin{lemma}
    $(\Gamma \text{ is consistent})$ implies ($S^*(S^*)$ does not halt).
    \label{lem:const-s}
\end{lemma}
\begin{proof}
    We can show that if $(\Gamma \text{ is consistent})$ is true then $S^*(S^*)$ must loop, since if $S^*(S^*)$ did halt there would be a proof $y_H$ showing as such, but then eventually our program would find $y_H$ in the line (for each x in Proofs:). Since $\Gamma$ is consistent there can be no other proof that $S^*(S^*)$ does not halt with lower index then $y_H$ the program to halt first. Since we started with the assumption that $S^*(S^*)$ halting and arrived at a contradiction then $S^*(S^*)$ must not halt.
\end{proof}

\begin{proof}[Proof of Godel 1]
For our statement we aim to show that there are no proofs for either $\Gamma \implies (S^*(S^*) \text{ halts})$ or $\Gamma \implies (S^*(S^*) \text{ doesn't halt})$

Since $\Gamma$ is consistent, we know that $S^*(S^*)$ does not halt, therefore using consistency again we know there can be no proof that it halts. 

We can say that there is no proof that it doesnt halt, since if there was a proof $y_L$ showing that it loops, then it would halt when the function iterated over $y_L$ contradicting our lemma \ref{lem:const-s}, and since it is consistent there is no other proof showing that $S^*(S^*)$ halts.  Thus no proof showing $S^*(S^*)$ doesn't halt exists.
\end{proof}

\begin{proof}[Proof of Godel 2]
To show that there is no proof $X$ in a system starting from the axioms and proving ts own consistency like so: $\Gamma \xRightarrow{X} (\Gamma \text{ is consistent})$. Since if there were, it would be possible to use $X$ with lemma \ref{lem:const-s} to construct a proof from $\Gamma$ implying that $S^*(S^*)$ loops, thus creating a contradiction with Godel 1. Thus there can be no way that a consistent set of axioms can prove their own consistency.
\end{proof}

