\chapter{Subobjects}



This is mainly helpful because it can help us generalize the concept of a subset to other categories, remember that for any subset $A \subset B$ we can define an injective function from $A \rightarrow B$.
\begin{definition}[Subobject]
    In a category $C$, a subobject $\varphi$ of $a$ is a monomorphism going from some object into $a$
    \[\begin{tikzcd}
	\text{dom}(\varphi) & a
	\arrow["\varphi", tail, from=1-1, to=1-2]
\end{tikzcd}\]
\end{definition}
\begin{example}
    A subset is a subobject in the category $Set$, a subgroup is a subobject in the category $Grp$, a subspace is a subobject in the category $Vect_\mathbb{R}$
\end{example}
Notice that without subobjects it is impossible to quantify most mathematical objects, the set of even natural numbers $\mathbb{N} \xhookrightarrow{f} \mathbb{N}$. \footnote{I cant figure out how to get the fancy tail working, so I am using a hook temporarially} Where we define $f(n)=2n$.

This also lets us talk generally about propositions on objects generally, where for any object $a$ the part of $a$ satisfying some proposition $P$ must be a subobject of $a$. 

\begin{definition}[Subobject Classifier]
    A subobject classifier in a category $\mathcal{C}$ is an object $\Omega$ along with a "Truth" arrow $\mathbb{1} \xrightarrow{\textbf{Truth}} \Omega$. Such that for every monomorphism $b \xhookrightarrow{\varphi} a$, then their exists a classification arrow $a \xrightarrow{\Chi_{\varphi}} \Omega$ such that the following diagram commutes:
    % https://q.uiver.app/#q=WzAsNCxbMCwxLCJhIl0sWzAsMCwiYiJdLFsxLDAsIlxcbWF0aGJiezF9Il0sWzEsMSwiXFxPbWVnYSJdLFsxLDAsIlxcdmFycGhpIiwyLHsic3R5bGUiOnsidGFpbCI6eyJuYW1lIjoibW9ubyJ9fX1dLFsxLDIsIiEiXSxbMiwzLCJcXHRleHRiZntUcnV0aH0iXSxbMCwzLCJcXGNoaV97XFx2YXJwaGl9IiwyXV0=
\[\begin{tikzcd}
	b & {\mathbb{1}} \\
	a & \Omega
	\arrow["\varphi"', tail, from=1-1, to=2-1]
	\arrow["{!}", from=1-1, to=1-2]
	\arrow["{\textbf{Truth}}", from=1-2, to=2-2]
	\arrow["{\chi_{\varphi}}"', from=2-1, to=2-2]
\end{tikzcd}\]
    and that when you fix $\chi_{\varphi} $ if you have another object $c$ with an arrow $c \xrightarrow[]{f} a$, such that $\chi_{\varphi} \circ f = \textbf{Truth} \circ \text{ }!$ , then there is a unique function $g$ that makes this diagram commmute
    % https://q.uiver.app/#q=WzAsNSxbMSwyLCJhIl0sWzEsMSwiYiJdLFsyLDEsIlxcbWF0aGJiezF9Il0sWzIsMiwiXFxPbWVnYSJdLFswLDAsImMiXSxbMSwwLCJcXHZhcnBoaSIsMix7InN0eWxlIjp7InRhaWwiOnsibmFtZSI6Im1vbm8ifX19XSxbMSwyLCIhIl0sWzIsMywiXFx0ZXh0YmZ7VHJ1dGh9Il0sWzAsMywiXFxjaGlfe1xcdmFycGhpfSIsMl0sWzQsMiwiISIsMCx7ImN1cnZlIjotMn1dLFs0LDAsImYiLDIseyJjdXJ2ZSI6Mn1dLFs0LDEsImciLDEseyJzdHlsZSI6eyJib2R5Ijp7Im5hbWUiOiJkYXNoZWQifX19XV0=
\[\begin{tikzcd}
	c \\
	& b & {\mathbb{1}} \\
	& a & \Omega
	\arrow["\varphi"', tail, from=2-2, to=3-2]
	\arrow["{!}", from=2-2, to=2-3]
	\arrow["{\textbf{Truth}}", from=2-3, to=3-3]
	\arrow["{\chi_{\varphi}}"', from=3-2, to=3-3]
	\arrow["{!}", curve={height=-12pt}, from=1-1, to=2-3]
	\arrow["f"', curve={height=12pt}, from=1-1, to=3-2]
	\arrow["g"{description}, dashed, from=1-1, to=2-2]
\end{tikzcd}\]
(The top diagram can be thoguht of as a way of associating every monomorphism/subobject of $a$ with an element of $Hom(a,\Omega)$, and the bottom diagram a way of associating every element of $Hom(a,\Omega)$ with a monomorphism/subobject)
\end{definition}
This definition looks really terse and that is an accurate assessment
\begin{quote}
    (The subobject classifier) is like a superdense nugget from outer space, and through scientific explorations in the latter half of the 20th century, we have found that it brings super powers to whichever categories possess it.
    -- David Spivak
\end{quote}
This is really useful since it allows us to quantify all subobjects or propositions depending on your interpretation as a single mathematical object $Hom(a, \Omega)$, this is especially useful in a cartesian closed category, where we can think of $\Omega^a$ as the internal object representing all subobjects of $a$.
\begin{definition}[Topos]
    A Topos is a category with
    \begin{enumerate}
        \item A terminal object and all products\footnote{And also equalizers}
        \item All exponentials
        \item A subobject classifier
    \end{enumerate}
\end{definition}
\begin{example}
    $Set$, $FinSet$, $Set^\mathbb{2}, DirGraph$, and depending on your definition $Cat$ are examples of a topos.
\end{example}
Following the example about the subobject giving a category superpowers, the following properties hold in any topos:
\begin{theorem}[Topos Properties]
    Every topos obeys the following rules and properties:
    \begin{enumerate}
        \item Has all sums (or in general all finite colimits)
        \item Sums and Product distribute nicely $(a + b) \times c = a \times c + b \times c$
        \item Every epi-monomorphism is an isomorphism
        \item Every morphism has epi mono factorizations, (essentially the categorical generalization of images)
        % https://q.uiver.app/#q=WzAsMyxbMCwwLCJhIl0sWzIsMCwiYiJdLFsxLDEsImltKGYpIl0sWzAsMSwiZiJdLFswLDIsIiIsMSx7InN0eWxlIjp7ImhlYWQiOnsibmFtZSI6ImVwaSJ9fX1dLFsyLDEsIiIsMSx7InN0eWxlIjp7InRhaWwiOnsibmFtZSI6Im1vbm8ifX19XV0=
        \[\begin{tikzcd}
        	a && b \\
        	& {im(f)}
        	\arrow["f", from=1-1, to=1-3]
        	\arrow[two heads, from=1-1, to=2-2]
        	\arrow[tail, from=2-2, to=1-3]
        \end{tikzcd}\]
        \item Every preorder of subobjects forms a heyting algebra.
        \item Every slice is cartesian closed
        \item Is regular and exact (Still figuring out what this means.)
    \end{enumerate}
\end{theorem}



This gives us a method of actually implementing most of the axioms of set theory namely:

Axiom of separation

Axiom of pairing

Axiom of union

Axiom of power set

These similar versions of these axioms can be used in any topos.


At this point we are almost at the point of fully modeling set theory inside of category theory:

\begin{definition}
    In any category $\mathcal{C}$ a natural numbers object is an object $\mathbb{N}$ combined with two arrows $\mathbb{N} \xrightarrow[]{succ} \mathbb{N}$ and $\mathbb{1} \xrightarrow[]{0} \mathbb{N}$ such that for any arrows $A \xrightarrow{f} A$, there exists an arrow $\mathbb{N} \xrightarrow{\sigma_f} A$ such that this diagram commutes:
        % https://q.uiver.app/#q=WzAsNSxbMCwwLCJcXG1hdGhiYnsxfSJdLFsxLDAsIlxcbWF0aGJie059Il0sWzIsMCwiXFxtYXRoYmJ7Tn0iXSxbMSwxLCJBIl0sWzIsMSwiQSJdLFswLDEsIjAiXSxbMCwzLCJnIiwyXSxbMyw0LCJmIiwyXSxbMSwyLCJzdWNjIl0sWzEsMywiXFxzaWdtYV9mIl0sWzIsNCwiXFxzaWdtYV9mIl1d
\[\begin{tikzcd}
	{\mathbb{1}} & {\mathbb{N}} & {\mathbb{N}} \\
	& A & A
	\arrow["0", from=1-1, to=1-2]
	\arrow["g"', from=1-1, to=2-2]
	\arrow["f"', from=2-2, to=2-3]
	\arrow["succ", from=1-2, to=1-3]
	\arrow["{\sigma_f}", from=1-2, to=2-2]
	\arrow["{\sigma_f}", from=1-3, to=2-3]
\end{tikzcd}\]
    You can think of this as a general way of expressing the fact that $\mathbb{N}$ is the smallest inductive set/object.
\end{definition}
\begin{theorem}[ETCS]
    Any category $\mathcal{C}$ that satisfies the following, has an internal logic identical to that of $Set$
    \begin{enumerate}
        \item $\mathcal{C}$ is a topos.
        \item $\mathcal{C}$ has a natural numbers object.
        \item $\mathcal{C}$ is "Well Pointed" (A categorical version of LEM and or the axiom of extensionality)
        \item For any epimorphism $a \xtwoheadrightarrow{f} b$, there exists an arrow $b \xrightarrow{\sigma} a$ such that $\sigma \circ f = id_a$. (Categorical version of the Axiom of Choice)
    \end{enumerate}
\end{theorem}

\begin{definition}[Elements]
    For any category $\mathcal{C}$ an element $e$ of an object $x$ is an arrow from $\mathbb{1}$ into $x$:
    \begin{align*}
        \mathbb{1} \xrightarrow[]{e} x
    \end{align*}
\end{definition}
\begin{definition}[Well Pointed]
    A category is well pointed if for any 2 arrows $f,g$ with domain $a \rightrightarrows b$, then $f=g$ if and only if for every element $\mathbb{1} \xrightarrow{x} a$. $f \circ x = g \circ x$
\end{definition}
Notice this can be thought of as a version of the law of excluded middle, consider the set of real numbers
\begin{align*}
    \{x \in \mathbb{R} | x>0 \vee x \leq 0\} \subseteq \mathbb{R}
\end{align*}
in a logic with the law of excluded middle we cannot prove inclusion the other way and thus it is a strict subset, however thinking about this from the perspective of elements we can see that there arent any elements of $\mathbb{R}$ that arent also in $\{x \in \mathbb{R} | x>0 \vee x \leq 0\} $, thus from the perspective of classical logic these "setlike" objects seem to have spacial properties that we would typically associate with topological spaces or manifolds.

\begin{theorem}
    Assuming the law of excluded middle holds in your meta theory, in any well pointed topos has boolean logic, aka $\Omega \cong \mathbb{1}+\mathbb{1}$
\end{theorem}
\begin{proof}
    This seems hard and a good exercise for me personally - Nicole
\end{proof}


\subsection{Exercises}

\begin{Exercise}
    \begin{tasks}
        \task In the category $Top$ of topological spaces and continuous functions. Show that "injective continuous functions" are monomorphisms. And that "surjective continuous functions" are epimorphisms.
        \task Read about \url{https://en.wikipedia.org/wiki/Space-filling_curve#History} for an example of an arrow that is both a monomorphism and an epimorphism but does not have a continuous inverse and is thus not an isomorphism.
    \end{tasks}
\end{Exercise}


\begin{Exercise}
    Decode the following commutative diagrams


    % https://q.uiver.app/#q=WzAsMyxbMSwwLCJcXG1hdGhiYntOfVxcdGltZXMgXFxtYXRoYmJ7Tn0iXSxbMSwxLCJcXG1hdGhiYntOfSJdLFswLDEsIlxcbWF0aGJie059Il0sWzAsMSwiKyJdLFsyLDEsImZzdCIsMl0sWzIsMCwiXFxsYW5nbGUgaWRfXFxtYXRoYmJ7Tn0sMCBcXHJhbmdsZSJdXQ==
\[\begin{tikzcd}
	& {\mathbb{N}\times \mathbb{N}} \\
	{\mathbb{N}} & {\mathbb{N}}
	\arrow["{+}", from=1-2, to=2-2]
	\arrow["fst"', from=2-1, to=2-2]
	\arrow["{\langle id_\mathbb{N},0 \rangle}", from=2-1, to=1-2]
\end{tikzcd}\]
% https://q.uiver.app/#q=WzAsNCxbMSwwLCJcXG1hdGhiYntOfSJdLFsxLDEsIlxcbWF0aGJie059Il0sWzAsMSwiXFxtYXRoYmJ7Tn1cXHRpbWVzIFxcbWF0aGJie059Il0sWzAsMCwiXFxtYXRoYmJ7Tn1cXHRpbWVzIFxcbWF0aGJie059Il0sWzIsMSwiKyIsMl0sWzEsMCwic3VjYyIsMl0sWzIsMywiXFxsYW5nbGUgaWRfXFxtYXRoYmJ7Tn0sIHN1Y2MgXFxyYW5nbGUiXSxbMywwLCIrIl1d
\[\begin{tikzcd}
	{\mathbb{N}\times \mathbb{N}} & {\mathbb{N}} \\
	{\mathbb{N}\times \mathbb{N}} & {\mathbb{N}}
	\arrow["{+}"', from=2-1, to=2-2]
	\arrow["succ"', from=2-2, to=1-2]
	\arrow["{ id_\mathbb{N} \times  succ }", from=2-1, to=1-1]
	\arrow["{+}", from=1-1, to=1-2]
\end{tikzcd}\]
\end{Exercise}




\begin{Exercise}
    Decode the following commutative diagram

    % https://q.uiver.app/#q=WzAsNCxbMSwwLCJcXG1hdGhiYntOfSBcXHRpbWVzIFxcbWF0aGJie059Il0sWzEsMSwiXFxtYXRoYmJ7Tn0iXSxbMCwxLCJcXG1hdGhiYntOfVxcdGltZXMgXFxtYXRoYmJ7MX0iXSxbMCwwLCJcXG1hdGhiYntOfVxcdGltZXN7XFxtYXRoYmJ7Tn19Il0sWzIsMSwiZnN0IiwyXSxbMCwxLCIqIl0sWzIsMywiXFxsYW5nbGUgaWRfXFxtYXRoYmJ7Tn0sMCBcXHJhbmdsZSJdLFszLDAsIlxcbGFuZ2xlIGlkX1xcbWF0aGJie059LHN1Y2MgXFxyYW5nbGUiXV0=
\[\begin{tikzcd}
	{\mathbb{N}\times{\mathbb{N}}} & {\mathbb{N} \times \mathbb{N}} \\
	{\mathbb{N}\times \mathbb{1}} & {\mathbb{N}}
	\arrow["fst"', from=2-1, to=2-2]
	\arrow["{*}", from=1-2, to=2-2]
	\arrow["{id_\mathbb{N} \times 0 }", from=2-1, to=1-1]
	\arrow["{id_\mathbb{N} \times succ }", from=1-1, to=1-2]
\end{tikzcd}\]
\end{Exercise}

