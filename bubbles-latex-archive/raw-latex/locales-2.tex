\chapter{Hine-Borel theorem for Locales}




Definitions 4.1. Given rational numbers a and b and a positive natural number n, a 2n-tuple (a 1,b 1,…,a n,b n) of rational numbers forms a zigzag of length n from a to b if we have:

    a=a 1, b=b n, and
    b i>a i+ for i=1,…,n−1.

A zigzag of positive length is orderly if we additionally have:

    a i<a i+ for i=1,…,n−1,
    a i<b i for i=1,…,n, and
    b i<b i+ for i=1,…,n−1.

As a technicality, if a≥b, then we also count the empty 0-tuple as an orderly zigzag of length 0 from a to b. (Notice that no zigzag of positive length can be orderly from a to b when a≥b.)

Given additionally a collection �� of opens, a zigzag (orderly or not) has zigs from �� if every pair (a i,b i) is contained in at least one member of ��.

Now we can replace any zigzag with an orderly zigzag with zigs from the same opens:

Lemma 4.2. (Zigzag Lemma)

Given rational numbers a and b, a natural number n, a collection �� of opens, and a zigzag from a to b of length n with zigs from ��, there exists an orderly zigzag from a to b of length at most n with zigs from ��.

Proof. If a≥b, then we use a zigzag of length 0. Otherwise, we assume that a<b and prove the lemma by induction on n. If n=1, then the original zigzag is orderly since a<b.

Now assume (as an inductive hypothesis) that the lemma holds for zigzags of length k for some k≥1. A zigzag of length k+ consists of a zig (a,b 1) from some open in ��, a zag b 1>a 2, and a zigzag of length k from a 2 to b with zigs from ��. Using the inductive hypothesis, replace the zigzag of length k with an orderly zigzag of length l≤k from a 2 to b=b l+ with zigs from ��. We now have a zigzag of length l+≤k+ from a to b with zigs from ��.

If a<a 2 and b 1<b 2, then also a<a 2<b 1, so this zigzag of length l+ from a to b is orderly.

Next suppose that a<a 2 but b 1≥b 2. In this case, consider the largest value of i≥2 such that b 1≥b i; if i=l+, then we can use the orderly zigzag of length 1 from a to b, which is contained in the same open as (a,b 1) was, since b 1≥b i=b. If i≤l instead, then take the orderly zigzag from a i+ to b, and precede it with the zig (a,b i), to get a zigzag of length l−i+2≤k from a to b. This zigzag is orderly, because a<a 2<a i+, and its zigs are from �� since (a,b i) is contained in the same open as (a,b 1) was (since b 1≥b i). So either way, we have an orderly zigzag of length at most k from a to b with zigs from ��.

Finally, suppose that a≥a 2. In this case, consider the largest value of i≥2 such that a≥a i, take the orderly zigzag from a i to b, and change a i to a to get a zigzag from a to b. If i≤l, then this is orderly since a<a i+<b i; if i=l+, then this is orderly since a<b=b i. Either way, (a,b i) belongs to the same open as (a i,b i) did (since a≥a i), so this orderly zigzag of length l−i+2≤k from a to b still has zigs from ��.

In any case, we have a zigzag of length at most k+ from a to b with zigs from ��, and the induction is complete.  ▮


Definition 8.1. An open cover of the unit interval is a collection �� of opens in the real line such that ℝ is the join of (−∞,0), (1,∞), and the members of ��.

Theorem 8.2. (Heine–Borel)

Every open cover of the unit interval has a finite subcover.

The proof is almost embarrassingly simple. The key point is that the construction of joins in terms of zigzags involves only finite zigzags, even for an infinitary join.

Proof. Let J be the join of (−∞,0), (1,∞), and the members of ��. Since this equals ℝ, then in particular (−2,3)⊆J, and since (−1,2)⋐(−2,3), we get a corresponding zigzag ζ involving finitely many zigs using finitely many of the members of ��. Let �� be the collection of these members of ��, and let K be the join of (−∞,0), (1,∞), and ��. Now if (a,b) is any pair of rational numbers, we construct a zigzag showing directly that (a,b)⊆K as follows: the zig (a,0)⊆(−∞,0), the zag 0>−1, the zigzag ζ from −1 to 2, the zag 2>1, and the zig (1,b)⊆(1,∞). This is always a valid zigzag, so K=ℝ. Therefore, the finite collection �� covers the unit interval.  ▮




\begin{lemma}
    
\end{lemma}