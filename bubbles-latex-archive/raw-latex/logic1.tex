
\chapter{Logic}
\begin{theorem}[Yoneda Lemma Consequence for Preorders]
    For any two objects $a,b\in \mathbb{X}$ then
    \begin{align*}
        a \rightsquigarrow b \text{ if and only if } \forall x \in \mathbb{X} \quad  x \rightsquigarrow a \implies x \rightsquigarrow b 
    \end{align*}
\end{theorem}

\begin{proof} ($\Rightarrow$)
Suppose $a \rightsquigarrow b$ and suppose that for some $x \in \mathbb{X}$, $x \rightsquigarrow a$. Then $x \rightsquigarrow b$ by transitivity. 

$(\Leftarrow)$ Suppose that $\forall x \in \mathbb{X}$, $x \rightsquigarrow a \implies x \rightsquigarrow b $. By reflexivity, $a \rightsquigarrow a$. Thus, $a \rightsquigarrow b$ by hypothesis.
\end{proof}
\begin{definition}[Lattice]
    A lattice is a partial order, where every 2 elements $a,b \in \mathbb{X}$ have a well defined infimim and supremum (called a join and meet respectively), defined to have the following properties

The meet of $a$ and $b$, denoted by $a \land b$, is an element of $\mathbb{X}$ such that:
\begin{enumerate}
\item $a \land b \rightsquigarrow a$
\item $a \land b \rightsquigarrow b$
\item For any $c \in \mathbb{X}$, if $c \rightsquigarrow a$ and $c \rightsquigarrow b$, then $c \rightsquigarrow a \land b$
\end{enumerate}
In other words, $a \land b$ is the greatest lower bound of $a$ and $b$. 

The join of $a$ and $b$, denoted by $a \lor b$, is defined the exact same way but all the arrows are reversed, namely it is an element of $\mathbb{X}$ such that:
\begin{enumerate}
\item $a \rightsquigarrow a \lor b$
\item $b \rightsquigarrow a \lor b$
\item For any $c \in \mathbb{X}$, if $a \rightsquigarrow c$ and $b \rightsquigarrow c$, then $a \lor b \rightsquigarrow c$
\end{enumerate}
In other words, $a \lor b$ is the least upper bound of $a$ and $b$.

These rules can be captured visually with the following commutative diagrams.
% https://q.uiver.app/#q=WzAsOCxbMCwxLCJhIl0sWzIsMSwiYiJdLFsxLDEsImEgXFx2ZWUgYiJdLFsxLDAsImMiXSxbMywxLCJhIl0sWzUsMSwiYiJdLFs0LDEsImFcXHdlZGdlIGIiXSxbNCwwLCJjIl0sWzAsMiwiIiwyLHsic3R5bGUiOnsiYm9keSI6eyJuYW1lIjoic3F1aWdnbHkifX19XSxbMSwyLCIiLDAseyJzdHlsZSI6eyJib2R5Ijp7Im5hbWUiOiJzcXVpZ2dseSJ9fX1dLFswLDMsIiIsMix7InN0eWxlIjp7ImJvZHkiOnsibmFtZSI6InNxdWlnZ2x5In19fV0sWzEsMywiIiwwLHsic3R5bGUiOnsiYm9keSI6eyJuYW1lIjoic3F1aWdnbHkifX19XSxbMiwzLCIiLDEseyJzdHlsZSI6eyJib2R5Ijp7Im5hbWUiOiJkb3R0ZWQifX19XSxbNyw2LCIiLDAseyJzdHlsZSI6eyJib2R5Ijp7Im5hbWUiOiJkYXNoZWQifX19XSxbNiw0LCIiLDAseyJzdHlsZSI6eyJib2R5Ijp7Im5hbWUiOiJzcXVpZ2dseSJ9fX1dLFs2LDUsIiIsMCx7InN0eWxlIjp7ImJvZHkiOnsibmFtZSI6InNxdWlnZ2x5In19fV0sWzcsNSwiIiwyLHsic3R5bGUiOnsiYm9keSI6eyJuYW1lIjoic3F1aWdnbHkifX19XSxbNyw0LCIiLDIseyJzdHlsZSI6eyJib2R5Ijp7Im5hbWUiOiJzcXVpZ2dseSJ9fX1dXQ==
\[\begin{tikzcd}
	& c &&& c \\
	a & {a \vee b} & b & a & {a\wedge b} & b
	\arrow[squiggly, from=2-1, to=2-2]
	\arrow[squiggly, from=2-3, to=2-2]
	\arrow[squiggly, from=2-1, to=1-2]
	\arrow[squiggly, from=2-3, to=1-2]
	\arrow[dotted, from=2-2, to=1-2]
	\arrow[dashed, from=1-5, to=2-5]
	\arrow[squiggly, from=2-5, to=2-4]
	\arrow[squiggly, from=2-5, to=2-6]
	\arrow[squiggly, from=1-5, to=2-6]
	\arrow[squiggly, from=1-5, to=2-4]
\end{tikzcd}\]
\end{definition}


\begin{definition}[Complete Lattice]
    A complete lattice is a lattice (aka a preorder with upper and lower bounds) with the additional property that there exist global upper and lower bounds like so. 
    \begin{enumerate}
        \item An upper bound $\textbf{False}$ such that for all $a \in \mathbb{X}$ then $\textbf{False} \rightsquigarrow a$
        \item An lower bound $\textbf{True}$ such that for all $a \in \mathbb{X}$ then $a \rightsquigarrow \textbf{True}$
    \end{enumerate}
(Jeremy could you throw in the definitions of these you wrote down in your notes, thanks!!!)
\end{definition}

\begin{definition}[Heyting Algebra]
    A heyting algebra is a complete lattice such that for any two objects $a,b$ there exists an object $(a \implies b)$, (also written as $b^a$ for reasons that are more understandable in cat theory), that satisfies the following
    \begin{enumerate}
        \item For all $a,b$ the following holds $(a \implies b) \wedge a \rightsquigarrow b$
        \item For any $c$ if $a \wedge c \rightsquigarrow b$, then $c \rightsquigarrow (a \implies b)$.
    \end{enumerate}
    We define $\neg a := (a \implies \textbf{False})$.
\end{definition}
\begin{adefinition}[Boolean Algebra]
    A boolean algebra is a heyting algebra $\mathbb{X}$ where for all $a \in \mathbb{X}$ then
    \begin{align*}
        a \vee \neg a \cong \textbf{True}.
    \end{align*}
\end{adefinition}
\begin{theorem}[Lindenbaum]
    A proposition $P$ holds in IPL (Classical logic without the law of excluded middle) if and only if $\textbf{Truth} \rightsquigarrow P$ in every heyting algebra. (Assuming you interpret joins and meets as their appropriate logical operators).
\end{theorem}
Interestingly the same theorem is both false and improvable in classical logic due to issues with godel's incompleteness theorem
\section{Exercises}
\begin{Exercise}
    In any Heyting Algebra show that $a \wedge \neg a \rightsquigarrow \textbf{False}$
\end{Exercise}
\begin{Answer}
    \begin{proof}
        Consider that $a \land \lnot a$ is equivalent to $a \land (a \implies \textbf{False})$ by definition of $\lnot a$. By modus ponens, we conclude $\textbf{False}$. 
    \end{proof}
\end{Answer}
\begin{Exercise}
    In any Heyting Algebra show that $a \cong (\textbf{Truth} \implies a)$ 
\end{Exercise}
\begin{Exercise}
    In any Heyting Algebra show that $a \rightsquigarrow b$ if and only if $\textbf{Truth} \rightsquigarrow (a \implies b)$
\end{Exercise}
\begin{Exercise}
In any Heyting Algebra show that:
    \begin{tasks}
        \task $a \vee \textbf{False} \cong a$
        \task $a \wedge \textbf{True} \cong a$
    \end{tasks}
\end{Exercise}
\begin{Exercise}
    Show that in any Heyting Algebra that $(a \implies b) \rightsquigarrow (\neg b \implies \neg a)$
\end{Exercise}
\begin{Answer}
    We aim to show that given $b \implies \textbf{False}$ and the implication $a \implies b$, then $a \implies \textbf{False}$. We begin with writing this out in full and using the uncurrying properties of implications
    \begin{align*}
         (b \implies \textbf{False}) &\implies (a \implies \textbf{False})\\
         (b \implies \textbf{False}) \wedge a &\implies \textbf{False}
    \end{align*}
    Now we can introduce our hypothesis $a \implies b$ to transform a simple composition since $a \implies b \implies \textbf{False}$ then $a \implies \textbf{False}$ and we are done.
\end{Answer}
\begin{Exercise}
    \begin{tasks}
        \task In any Heyting Algebra show that $\neg a \vee b \rightsquigarrow a \implies b$
        \task Prove that in a boolean algebra under the alternative definition that $(a \implies b) \rightsquigarrow \neg a \vee b$
    \end{tasks}
\end{Exercise}
\begin{Answer}
    \begin{tasks}
        \task Consider that under distributivity we have \begin{align*}
            a \wedge (\neg a \vee b) &\cong (a \wedge \neg a) \vee (a \wedge b)\\
            &\cong \textbf{False} \vee (a \wedge b)\\
            &\cong a \wedge b \rightsquigarrow b
        \end{align*}
        By the definition of $a \implies b$ we have $\neg a \vee b \rightsquigarrow a \implies b$
        \task  
    \end{tasks}
\end{Answer}
\begin{Exercise}
    Show that in any Heyting Algebra that $\neg( \neg (A \wedge \neg A))$ holds.
\end{Exercise}
\begin{Answer}
    To show that $\neg( \neg (A \wedge \neg A))$ is true, we first assume that $\neg( A \wedge \neg A)$ is false and try to arrive at a contradiction, namely by showing $A \wedge \neg A$ creating a contradiction by \ref{ex}

    Let us attempt to prove $\neg A$, to do this we first assume $A$ is true and arrive at a contradiction, if $A$ is true then $A \wedge \neq A$ is true, arriving at a contradiction since we are assuming $\neg( A \wedge \neg A)$. Therefore $\neg A$ is true under the assumption $\neg( A \wedge \neg A)$, however $\neg A \implies A \wedge \neq A$, thus creating a contradiction with $\neg( A \wedge \neg A)$. Thus showing that $\neg( \neg (A \wedge \neg A))$ is true.
\end{Answer}

\begin{Exercise}
    \begin{tasks}
        \task Show that in any Heyting Algebra $a \rightsquigarrow \neg(\neg a)$
        \task Show that in any Heyting Algebra $\neg a \rightsquigarrow \neg (\neg (\neg a))$
    \end{tasks}
\end{Exercise}
\begin{Answer}
    \begin{proof}
        To prove $a \implies \neg(\neg a)$ we should assume $\neg a$ and aim for a contradiction. However since we know that $a$ is true, then $a \wedge \not a$ gives us a contradiction. Thus $a \implies \neg(\neg a)$.
    \end{proof}
    \begin{proof}
        By definition we have that $\neg (\neg(\neg a))= (\neg(\neg a)) \rightsquigarrow \textbf{False}$
    \end{proof}
\end{Answer}




