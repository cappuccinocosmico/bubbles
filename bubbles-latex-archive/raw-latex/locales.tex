\chapter{Locales}

\begin{definition}[Frame]
    A frame (also sometimes confusingly refered to as a "locale") is a partial order with:
    \begin{itemize}
        \item All (potentially infinite) joins
        \item Finite meets
    \end{itemize}
    Note that this implies the existence of upper and lower bounds, since the lower bound is the meet of zero elements, and the upper bound is the join of all elements in your frame.
\end{definition}
\begin{definition}[Frame Homomorphism]
    A frame homomorphism is an arrow from frames $A \xrightarrow[]{f} B$. Such that
    \begin{align*}
        f\left(\bigvee_{\alpha \in I} x_\alpha \right) &= \bigvee_{\alpha \in I} f(x_\alpha) \\
        f\left(\bigwedge_{i=1}^n x_i \right) &= \bigwedge_{i=1}^n f(x_i) \\
    \end{align*}
\end{definition}

\begin{definition}[Topological Space]
    A topological space $|X|$ is a set with a frame of opens $\mathcal{O}(X)$, such that $\mathcal{O}(X)$ is a subframe with the same meets and joins as the partial order $\mathcal{P}(|X|)$ under the subset relation. (Notice that the lowest bound is $\phi$ and the upper bound is $X$ since the Frame must have an upper and lower bound and that must match the bounds of $\mathcal{P}(X)$)
\end{definition}
\begin{definition}[Continuous function in $Top$]
    A continuous function between topologies $X \xrightarrow[]{f} Y$ is a function from the sets $|X| \xrightarrow[]{f} |Y|$ such that the preimage of open sets $\mathcal{O}(Y) \xrightarrow[]{f^{-1}(\_)} \mathcal{O}(X)$ is a frame homomorphism.
\end{definition}
\begin{definition}[Elements of a Topology]
    Elements of a topology can be formalized in 2 ways, an element $a$ of $X$ can be thought of as an element of $a \in |X|$, or alternatively
    as a homomorphism from the topological space $\mathcal{P}(\mathbb{1}) \xrightarrow[]{a} X$, where taking the singular elment $\star$ in $\mathbb{1}$ then the continuous function equals: $a(\star) =a \in |X|$
\end{definition}
\begin{definition}[Sober Topology]
    A meet irriducible set in topological space $(X, \tau)$ is a set $Y$ such that for any 2 opens $U,V$, then if there exists an element $x \in \mathbb{X}$ and  $U \cap V \subseteq X \setminus \{x\}$, then either $U \subseteq X \setminus \{x\}$ or $V \subseteq X \setminus \{x\}$ 
\end{definition}
\begin{theorem}
    Every hausdoff space is sober.
\end{theorem}
\begin{proof}
    See Exercises
\end{proof}

\begin{definition}[Locales and Continuous Function]
    A locale $Y$ "is" a frame opens $\mathcal{O}(Y)$, and a continuous function between locales $X \xrightarrow[]{f} Y$ is exactly a frame homomorphism $f^*$ from $\mathcal{O}(Y) \xrightarrow[]{f^*} \mathcal{O}(X)$.
\end{definition}
\begin{definition}[Elements of a Locale]
    Elements of a locale are a continuous function from the locale $\mathcal{P}(\mathbb{1}) \xrightarrow[]{a} X$. However, to further motivate this comparision let us remember that from the subspace classifier chapter that $\mathcal{P}(1) \cong \Omega^\mathbb{1} \cong \Omega$. Remember, that one can instead of viewing set inclusion as a primative operator, you can think of it as a function that takes in a set and an element, and returns a truth value dependant 
\end{definition}
\begin{definition}[Compact Locale]
    A locale $L$ is compact if and only if whenever an arbitrary set of open has a join of the maximal element, there is a finite subset of those opens that also 
\end{definition}
\begin{definition}[Disconnected Locale]
    A locale $X$ is disconnected if there are 2 opens  $U,V \in \mathcal{O}(X)$, such that $U \wedge V = \phi$ and $U \vee V = X$. (In terms of a topological space $X$, a topological space is disconnected if there are 2 disjoint open sets, $U,V$ such that $U \cup V = X$)
\end{definition}


\begin{definition}[$L$-Functor]
    We define a functor $L$ to be a functor that takes every topological space, and associates it with the locale representing its frame of opens, and associates every continuous function with the continuous map on locale's corresponding to it's preimage map
\end{definition}

\begin{definition}
    
\end{definition}



From a classic topology book the topology on the real numbers is often defined like so
\begin{lie}[Classic topology on the Reals V1]
    A subset $U$ of the real line is a member of $\mathcal{O}(\mathbb{R})$ if and only if $U$ can be written as an arbitrary union of open intervals.
\end{lie}
However notice that if you give me any real interval like $(-\pi,\pi)$ I can write it as an infinite union of rational intervals like so:
\begin{align*}
    U_0 &= (-3,3)\\
    U_1 &= (-3.1,3.1)\\
    U_2 &= (-3.14,3.14)\\
    U_3 &= (-3.141,3.141)\\
    & \vdots\\
    \bigcup_{n=0}^\infty U_n &= (-\pi,\pi)
\end{align*}
Notice that we can actually redefine the real line with a stronger condition.
\begin{lie}[Classic topology on the Reals V2]
    A subset $U$ of the real line is a member of $\mathcal{O}(\mathbb{R})$ if and only if $U$ can be written as an arbitrary union of open intervals with rational values.
\end{lie}

\begin{theorem}
    The rational numbers under the standard topology are disconnected at every irrational number.
\end{theorem}
\begin{proof}
    Consider an irrational number $x$, then consider a monotonic increasing sequence of rational numbers $v_n \mapsto x$ converging to $x$. And a monotonic decreasing sequence $w_n \mapsto x$ also converging to $x$. Then
    \begin{align*}
        \mathbb{Q} &= ((-\infty,x)\cap \mathbb{Q}) \cup ((x,\infty)\cap \mathbb{Q})\\
        &=\left(\bigcup_{n\in \mathbb{N}} (-\infty,v_n) \right) \cup \left(\bigcup_{n\in \mathbb{N}} (w_n, \infty) \right)
    \end{align*}
\end{proof}

This definition of the real numbers is taken from toby bartel's page on the nlab
\begin{definition}[The localic definition of the $\mathbb{R}$eal Line]
    We consider the real line to be a subframe of $\mathcal{P}(\mathbb{Q} \times \mathbb{Q})$, we say that an element $X \in \mathcal{P}(\mathbb{Q} \times \mathbb{Q})$ (X can be thought of as a relation on $\mathbb{Q}$)

    To help leverage some existing intuitions about subsets of intervals, let us say that $(a,b) \in X$ and $(c,d) \in X$. Then we say that
    \begin{align*}
        (a,b) \subseteq (c,d) &:= (c \leq a) \wedge (b \leq d)\\ 
        (a,b) \subsetneq (c,d) &:= (c < a) \wedge (b < d)
    \end{align*}
    From here we can say that $X \in \mathcal{O}(X)$ if and only if it meets the following criterions. For the sake of notational convinence I have tried to define all the definitions below such that $a \leq b \leq c \leq d$
    \begin{enumerate}
        \item If $a \leq b$, then $(b,a) \in X$ (Aka, all rational intervals that are empty are contained in $X$)
        \item If $(b,c) \subseteq (a,d)$ and $(a,d) \in X$ then $(b,c) \in X$ (Aka inclusion is transitive)
        \item If $(a,c) \in X$ and $(b,d) \in X$ and $b \leq c$ then $(a,d) \in X$. (Aka, the union of 2 overlapping intervals is also a open interval.)
        \item If there exist $a,d \in \mathbb{Q}$, such that for every $(b,c) \subsetneq (a,dd)$ then $(b,c) \in X$, then that implies $(a,d) \in X$
    \end{enumerate}
\end{definition}


\subsection{Exercises}

\begin{Exercise}
    A topological space $(X,\tau)$ is hausdorff if and only if for every $x,y \in X$, then there exist opens $U,V$ such that $x \in U$ and $y \in V$ and $U \cap V = \phi$.

    Show that every hausdorff space is sober.
\end{Exercise}