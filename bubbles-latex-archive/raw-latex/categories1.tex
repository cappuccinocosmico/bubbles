\chapter{Categories I: Categories and their Examples}
\section{Definitions}
\begin{definition*}[Category]
    A Category $C$ consists of a class of objects $\mathcal{O}$, and a set of arrows $Hom_C(a,b)$ for every ordered pair of objects $a,b \in \mathcal{O}$. There exist certain arrows
    \begin{enumerate}
        \item For every object $a$ there exists an arrow $id_a \in Hom_C(a,a)$
        \item For any arrow $f \in Hom_C(a,b)$ and $g \in Hom_C(b,c)$, there exists an arrow $g \circ f \in Hom_C(a,c)$
        
    \end{enumerate}
    These arrows must satisfy certain properties namely:
    \begin{enumerate}
        \item For any morphism $a \xrightarrow{f} b$ then $b_{id} \circ f = f = f \circ id_a$. (ie, the identity arrow must behave like an identity function or map on sets or any other mathematical object.)
        % https://q.uiver.app/#q=WzAsNCxbMCwwLCJhIl0sWzEsMCwiYSJdLFsyLDAsImIiXSxbMywwLCJiIl0sWzAsMSwiYV97aWR9Il0sWzEsMiwiZiJdLFsyLDMsImJfe2lkfSJdLFswLDIsImYiLDIseyJjdXJ2ZSI6Mn1dLFsxLDMsImYiLDIseyJjdXJ2ZSI6Mn1dXQ==
        \[\begin{tikzcd}
            a & a & b & b
            \arrow["{id_a}", from=1-1, to=1-2]
            \arrow["f", from=1-2, to=1-3]
            \arrow["{b_{id}}", from=1-3, to=1-4]
            \arrow["f"', curve={height=12pt}, from=1-1, to=1-3]
            \arrow["f"', curve={height=12pt}, from=1-2, to=1-4]
        \end{tikzcd}\]
        \item For any $f \in Hom_C(a,b)$, $g \in Hom_C(b,c)$ and $h \in Hom_C(b,c)$, then $h \circ(g \circ f)=(h \circ g) \circ f$ (ie, composition of arrows must be associative like functions or maps on sets or any other mathematical object.)
        % https://q.uiver.app/#q=WzAsNCxbMCwwLCJhIl0sWzEsMCwiYiJdLFsyLDAsImMiXSxbMywwLCJkIl0sWzAsMSwiZiJdLFsxLDIsImciXSxbMiwzLCJoIl0sWzAsMiwiZyBcXGNpcmMgZiIsMix7ImN1cnZlIjoyfV0sWzEsMywiaFxcY2lyYyBnIiwyLHsiY3VydmUiOjJ9XSxbMCwzLCJoIFxcY2lyYyBnIFxcY2lyYyBmIiwwLHsiY3VydmUiOi00fV1d
        \[\begin{tikzcd}
            a & b & c & d
            \arrow["f", from=1-1, to=1-2]
            \arrow["g", from=1-2, to=1-3]
            \arrow["h", from=1-3, to=1-4]
            \arrow["{g \circ f}"', curve={height=12pt}, from=1-1, to=1-3]
            \arrow["{h\circ g}"', curve={height=12pt}, from=1-2, to=1-4]
            \arrow["{h \circ g \circ f}", curve={height=-24pt}, from=1-1, to=1-4]
        \end{tikzcd}\]
    \end{enumerate}
\end{definition*}
\begin{definition*}[Isomorphism]
    Two objects $a,b$ in a category $\mathcal{C}$ are isomorphic if and only if there exists an arrow $a \xrightarrow[]{f} b$ and an arrow $b \xrightarrow[]{f^{-1}} a$ such that $f \circ f^{-1} = id_b$ and $f^{-1} \circ f = id_a$. (This is often written as $a \cong b$)
\end{definition*}
\begin{definition}[Monomorphism]
    A Monomorphism in a category $\mathcal{C}$ is an arrow $x$ from $a \xrightarrow[]{x} b$, such that for every $c$ and arrows $c \xrightarrow[]{f} a$ and $c \xrightarrow[]{g} a$. 
    % https://q.uiver.app/#q=WzAsMyxbMCwwLCJjIl0sWzEsMCwiYSJdLFsyLDAsImIiXSxbMCwxLCJmIiwwLHsib2Zmc2V0IjotMX1dLFswLDEsImciLDIseyJvZmZzZXQiOjF9XSxbMSwyLCJ4Il1d
\[\begin{tikzcd}
	c & a & b
	\arrow["f", shift left, from=1-1, to=1-2]
	\arrow["g"', shift right, from=1-1, to=1-2]
	\arrow["x", from=1-2, to=1-3]
\end{tikzcd}\]
    
    Then $ x \circ f = x \circ g$ if and only if $f=g$. These are important enough that if $f$ is a monomorphism it is drawn like:
    \[\begin{tikzcd}
	a & b
	\arrow["x", tail, from=1-1, to=1-2]
\end{tikzcd}\]
\end{definition}
\begin{definition}[Epimorphism]
    This is just the dual of a monomorphism, namely an arrow $a \xrightarrow[]{x} b$, such that for every  $b \xrightarrow[]{f} c$ and $b \xrightarrow[]{g} c$. 
    % https://q.uiver.app/#q=WzAsMyxbMCwwLCJhIl0sWzEsMCwiYiJdLFsyLDAsImMiXSxbMCwxLCJ4IiwwLHsic3R5bGUiOnsiaGVhZCI6eyJuYW1lIjoiZXBpIn19fV0sWzEsMiwiZiIsMCx7Im9mZnNldCI6LTF9XSxbMSwyLCJnIiwyLHsib2Zmc2V0IjoxfV1d
\[\begin{tikzcd}
	a & b & c
	\arrow["x", two heads, from=1-1, to=1-2]
	\arrow["f", shift left, from=1-2, to=1-3]
	\arrow["g"', shift right, from=1-2, to=1-3]
\end{tikzcd}\]
Then $ f \circ x = g \circ x$ if and only if $f=g$.
\end{definition}
We can think of monomorphisms as categorical generalization of an "injective function". (And likewise epimorphisms as categorical generalizations of a "surjective function").


\begin{definition*}[Preorder]
    A preorder $\mathbb{X}$ is a set with a relation $a \rightsquigarrow b$, that satisfies
    \begin{enumerate}
        \item For all $a \in \mathbb{X}$ then $a \rightsquigarrow a$.
        \item If $a \rightsquigarrow b$ and $b \rightsquigarrow c$, then $a \rightsquigarrow c$
    \end{enumerate}
    We say that $a \cong b$ if and only if $a \rightsquigarrow b$ and $b \rightsquigarrow a$
\end{definition*}
\begin{definition*}[Partial Order Alternate]
    A partial order is a preorder where $a \cong b$ if and only if $a=b$
\end{definition*}
\begin{definition*}[Preorder Alternate]
    A Preorder is either 
    \begin{itemize}
        \item A category with the special property that for any 2 arrows between identical objects, $a \xrightarrow{f} b$ and $a \xrightarrow{g} b$, then $g=f$.
        \item A category where there is either one or zero arrows between any two objects. (This is referred to as a category "enriched" over $\{\mathbb{0},\mathbb{1}\}$, a exact definition is given later.) 
    \end{itemize}
    
\end{definition*}

%\newpage
\section{Exercises}
\begin{Exercise}
    Is a partial order just a skeletal preorder?
    
    Short answer, yes. I will take a look at each of the three definitions of preorder given above. First is the set theoretic one, where a preorder $P$ on a set $\mathbb{X}$ is a set $P \subseteq \mathbb{X} \times \mathbb{X}$ s.t $((a,b) \in P \land (b,c) \in P)\Rightarrow (a,c) \in P$ and $\forall a (a \in \mathbb{X} \Rightarrow (a,a) \in P)$. We may write $a \rightsquigarrow b$ whenever $(a,b) \in P$. 
    
    Furthermore,  if for $a,b \in \mathbb{X}$ it is the case that $(a \rightsquigarrow b) \land (b \rightsquigarrow a)$ then we write $a \cong b$. 

    A partial order $P'$ on a set $\mathbb{X}$ is a preorder with the additional property that for all $a,b \in \mathbb{X},  a\cong b \iff a = b$.\footnote{I will assume that the meaning of `=' will vary depending upon the context.} This makes our ears perk up, as a skeltal category $C$ is a category in which two objects are isomorphic iff they're equal. 

    
\end{Exercise}

\begin{Exercise}
    \begin{enumerate}[label=(\alph*)]
        \item Show that $id_a$ is unique, aka if there are two arrows $a \xrightarrow[]{f} a$ and $a \xrightarrow[]{g} a$ where both satisfy the properties of the identity arrow, namely:
         % https://q.uiver.app/#q=WzAsNixbMCwxLCJhIl0sWzAsMCwiYSJdLFsxLDEsImIiXSxbMiwwLCJiIl0sWzIsMSwiYSJdLFszLDEsImEiXSxbMiwwLCJ4Il0sWzAsMSwiaWRfYSJdLFsyLDEsIngiLDJdLFs1LDQsImlkX2EiXSxbNSwzLCJ5IiwyXSxbNCwzLCJ5Il1d
        \[\begin{tikzcd}
        	a && b \\
        	a & b & a & a
        	\arrow["x", from=2-2, to=2-1]
        	\arrow["{id_a}", from=2-1, to=1-1]
        	\arrow["x"', from=2-2, to=1-1]
        	\arrow["{id_a}", from=2-4, to=2-3]
        	\arrow["y"', from=2-4, to=1-3]
        	\arrow["y", from=2-3, to=1-3]
        \end{tikzcd}\]
        Show that $f=g$
        \item Assuming that there exists some arrow $a \xrightarrow[]{y} a$ such that $F(y)=id_{F(a)}$ show that $F(id_a)=id_{F(a)}$ using only the fact that $F(f \circ g)= F(f) \circ F(g)$.
        
    \end{enumerate}

\end{Exercise}


\begin{Answer}
    \begin{enumerate}[label=(\alph*)]
        \item Consider our two arrows $f,g$ that satisfy the properties of an identity arrow. Now using the fact that $f$ is an identity arrow we have $f \circ g=g$, and since $g$ is an identity arrow we have $f \circ g =f$. Therefore $g=f$
        \item Consider that 
        \begin{align*}
            F(y \circ id_a) &= F(y)\\
            F(y \circ id_a) &= id_{F(a)}\\
            F(y) \circ F(id_a) &= id_{F(a)}\\
            id_{F(a)} \circ F(id_a) &= id_{F(a)}\\
            F(id_a) &= id_{F(a)}
        \end{align*}
    \end{enumerate}
\end{Answer}


\begin{Exercise}
    Prove that if an arrow is an isomorphism it is also a monomorphism and an epimorphism. 

    \begin{proof}

        Let $\mathcal{C}$ be a category and suppose that $a,b, \text{ and }c$ are objects in $\mathcal{C}$. Let $x \in Hom(a,b)$ be an isomorphism and suppose that $f,g \in Hom(b,c)$ for some object $c$ like so 



        % https://q.uiver.app/#q=WzAsMyxbMiwwLCJiIl0sWzQsMCwiYyJdLFswLDAsImEiXSxbMCwxLCJnIiwyLHsib2Zmc2V0IjoyfV0sWzAsMSwiZiIsMCx7Im9mZnNldCI6LTJ9XSxbMiwwLCJ4Il1d
        \[\begin{tikzcd}
        	a && b && c
        	\arrow["g"', shift right=2, from=1-3, to=1-5]
        	\arrow["f", shift left=2, from=1-3, to=1-5]
        	\arrow["x", from=1-1, to=1-3]
        \end{tikzcd}\]

        I aim to show that $x$ is an epimorphism. Toward this end, suppose that $f \circ x = g\circ x$ and note that since $x$ is an isomorphism there exists an arrow $x^{-1} \in Hom(b,a)$\footnote{Is it always the case that arrows in a category comprise a set? - Jeremy} such that  $x^{-1} \circ x = id_a$ and $x \circ x^{-1} = id_b$. Now, 

        \begin{align*}
        f \circ x &= g\circ x \\
        f \circ x \circ x^{-1} &= g \circ x \circ x^{-1} \\
        f \circ id_b &= g\circ id_b \quad  \\ 
        f &= g \quad 
        \end{align*} 

        Conversely, if $f = g$ then 

        \begin{align*}
        f \circ id_b &= g\circ id_b \quad  \\ 
        f \circ x \circ x^{-1} &= g \circ x \circ x^{-1}\quad  \\ 
        f \circ x \circ x^{-1} \circ x &= g \circ x \circ x^{-1} \circ x \\
        f \circ x \circ id_a &= g\circ x \circ id_a  \\
        f \circ x &= g\circ x 
        \end{align*} 

        Hence, $x$ is an epimorphism. 

        To see that $x$ a monomorphism, suppose that $k, l \in Hom(c,a)$ like so 

        % https://q.uiver.app/#q=WzAsMyxbMCwwLCJjIl0sWzIsMCwiYSJdLFs0LDAsImIiXSxbMCwxLCJsIiwyLHsib2Zmc2V0IjoyfV0sWzAsMSwiayIsMCx7Im9mZnNldCI6LTJ9XSxbMSwyLCJ4Il1d
        \[\begin{tikzcd}
	       c && a && b
	       \arrow["l"', shift right=2, from=1-1, to=1-3]
	       \arrow["k", shift left=2, from=1-1, to=1-3]
	       \arrow["x", from=1-3, to=1-5]
        \end{tikzcd}\]
        
        and suppose that that $x\circ k = x \circ l $. Then 

        \begin{align*}
        x^{-1} \circ x\circ k &=  x^{-1} \circ x\circ l \\ 
        id_a \circ k &= id_a \circ l \\
        k &= l
        \end{align*} 

        Conversely, if $k = l$ then 

        \begin{align*}
        id_a \circ k & = id_a \circ l \\
        x^{-1} \circ x \circ k & = x^{-1} \circ x \circ l \\
        x \circ x^{-1} \circ x \circ k & = x\circ x^{-1} \circ x \circ l \\
        id_b \circ x \circ k & = id_b \circ x \circ l \\
        x \circ k & =  x \circ l
        \end{align*}

        Hence, $x$ is monic. 
    \end{proof}
\end{Exercise}