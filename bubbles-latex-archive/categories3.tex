\chapter{Categories III: Natural Transformations and 2-Categories}
\begin{definition}[Nat Transformation]
    A Natural transformation $\mu$ between two functors $A \xrightarrow{F} B$ and $A \xrightarrow{G} B$, associates every object $x \in A$ with an arrow in $B$, $\mu_x$ with the following domain and codomain
    \[F(x) \xrightarrow{\mu_x} G(X)\]
    such that for every arrow in $x \xrightarrow{f} y \in A$ then the $\mu_y \circ F(f) = G(f) \circ \mu_x$ or alternatively the following diagram commutes.
    % https://q.uiver.app/#q=WzAsNCxbMCwwLCJGKHgpIl0sWzAsMSwiRih5KSJdLFsxLDAsIkcoeCkiXSxbMSwxLCJHKHkpIl0sWzAsMSwiRihmKSIsMl0sWzAsMiwiXFxtdV94Il0sWzEsMywiXFxtdV95IiwyXSxbMiwzLCJHKGYpIl1d
\[\begin{tikzcd}
	{F(x)} & {G(x)} \\
	{F(y)} & {G(y)}
	\arrow["{F(f)}"', from=1-1, to=2-1]
	\arrow["{\mu_x}", from=1-1, to=1-2]
	\arrow["{\mu_y}"', from=2-1, to=2-2]
	\arrow["{G(f)}", from=1-2, to=2-2]
\end{tikzcd}\]
\end{definition}
\begin{definition}[Nat Isomorphism]
    A natural isomorphism is a natural transformation $\mu$ where every morphism $\mu_x$ is a bijection/isomorphism.
\end{definition}
\begin{definition}
    Two categories are $A$ and $B$ are equivalent if there exist functors $A \xrightarrow{F} B$ and $B \xrightarrow{G} A$ such that there is a natural isomorphism between $F \circ G$ and $A_{id}$ as well as a natural isomorphism between $G \circ F$ and $B_{id}$
    % https://q.uiver.app/#q=WzAsNixbMSwwLCJcXG1hdGhjYWx7QX0iXSxbMiwwLCJcXG1hdGhjYWx7Qn0iXSxbMiwxLCJcXG1hdGhjYWx7QX0iXSxbMywxLCJcXG1hdGhjYWx7QX0iXSxbMSwxLCJcXG1hdGhjYWx7Qn0iXSxbMCwxLCJcXG1hdGhjYWx7Qn0iXSxbMCwxLCJGIiwwLHsiY3VydmUiOi0xfV0sWzEsMCwiRyIsMCx7ImN1cnZlIjotMX1dLFsyLDMsIkcgXFxjaXJjIEYiLDAseyJjdXJ2ZSI6LTJ9XSxbMywyLCJpZF9cXG1hdGhjYWx7QX0iLDAseyJjdXJ2ZSI6LTJ9XSxbNSw0LCJGIFxcY2lyYyBHIiwwLHsiY3VydmUiOi0yfV0sWzQsNSwiaWRfXFxtYXRoY2Fse0J9IiwwLHsiY3VydmUiOi0yfV0sWzksOCwiIiwwLHsic2hvcnRlbiI6eyJzb3VyY2UiOjIwLCJ0YXJnZXQiOjIwfSwic3R5bGUiOnsidGFpbCI6eyJuYW1lIjoiYXJyb3doZWFkIn19fV0sWzExLDEwLCIiLDAseyJzaG9ydGVuIjp7InNvdXJjZSI6MjAsInRhcmdldCI6MjB9LCJzdHlsZSI6eyJ0YWlsIjp7Im5hbWUiOiJhcnJvd2hlYWQifX19XV0=
\[\begin{tikzcd}
	& {\mathcal{A}} & {\mathcal{B}} \\
	{\mathcal{B}} & {\mathcal{B}} & {\mathcal{A}} & {\mathcal{A}}
	\arrow["F", curve={height=-6pt}, from=1-2, to=1-3]
	\arrow["G", curve={height=-6pt}, from=1-3, to=1-2]
	\arrow[""{name=0, anchor=center, inner sep=0}, "{G \circ F}", curve={height=-12pt}, from=2-3, to=2-4]
	\arrow[""{name=1, anchor=center, inner sep=0}, "{id_\mathcal{A}}", curve={height=-12pt}, from=2-4, to=2-3]
	\arrow[""{name=2, anchor=center, inner sep=0}, "{F \circ G}", curve={height=-12pt}, from=2-1, to=2-2]
	\arrow[""{name=3, anchor=center, inner sep=0}, "{id_\mathcal{B}}", curve={height=-12pt}, from=2-2, to=2-1]
	\arrow[shorten <=3pt, shorten >=3pt, Rightarrow, 2tail reversed, from=1, to=0]
	\arrow[shorten <=3pt, shorten >=3pt, Rightarrow, 2tail reversed, from=3, to=2]
\end{tikzcd}\]
\end{definition}
\begin{theorem}
    All "categorical" constructions, any definition of a sets/object by only referencing their behavior with respect to functions/arrows, (aka no subsets, no inclusion, no equality on objects) respect equivalence of categories, the same way all operations in group theory respect isomorphism of groups and the like. 
\end{theorem}


\begin{definition}[Functor Category]
    \label{def:funccat}
    For any two categories $\mathcal{A},\mathcal{B}$, we define the category $\mathcal{B}^\mathcal{A}$ to be the category of functors from $\mathcal{A}$ to $\mathcal{B}$, where
    \begin{itemize}
        \item Objects are functors from $\mathcal{A}$ to $\mathcal{B}$
        \item Arrows from objects $F$ to $G$ are natural transformations from a functor $F$ to a functor $G$
    \end{itemize}
    We define
    \begin{itemize}
        \item For any functor $F$ we define the natural transformation $F \xRightarrow{id_F} F$ to be the natural transformation associating every object $x \in \mathcal{A}$ with the arrow in $\mathcal{B}$
        \begin{align*}
            F(x) \xrightarrow[]{id_{F(x)}} F(x)
        \end{align*}
        \item Given three functors $F,G,H$ between $\mathcal{A}$ and $\mathcal{B}$, and natural transformations $F \xRightarrow{\alpha} G$ and $G \xRightarrow{\beta} H$. The natural transformation $\beta \circ \alpha$ associates every $x \in \mathcal{A}$ with the arrow in $\mathcal{B}$ given by:
        % https://q.uiver.app/#q=WzAsMyxbMCwwLCJGKHgpIl0sWzEsMCwiRyh4KSJdLFsyLDAsIkgoeCkiXSxbMCwxLCJcXGFscGhhX3giXSxbMSwyLCJcXGJldGFfeCJdLFswLDIsIihcXGJldGEgXFxjaXJjIFxcYWxwaGEpX3g9XFxiZXRhX3ggXFxjaXJjIFxcYWxwaGFfeCIsMix7ImN1cnZlIjoyfV1d
\[\begin{tikzcd}
	{F(x)} & {G(x)} & {H(x)}
	\arrow["{\alpha_x}", from=1-1, to=1-2]
	\arrow["{\beta_x}", from=1-2, to=1-3]
	\arrow["{(\beta \circ \alpha)_x=\beta_x \circ \alpha_x}"', curve={height=12pt}, from=1-1, to=1-3]
\end{tikzcd}\]
        
    \end{itemize}
    Proofs that the definitions for identity and composition exist as natural transformations and satisfy the rules of a category are in the exercise solutions.
\end{definition}

\

\begin{definition}[2-Category]
    A 2-Category $C$ consists of a collection of objects, a collection of 1-morphisms $Hom_C(a,b)$ for every ordered pair of objects $a,b \in \mathcal{O}$. And a collection $Hom2(f,g)$ for every ordered pair of arrows $f,g \in Hom_C(a,b)$
    \begin{enumerate}
        \item For every object $a$ there exists a 1-morphism $id_a \in Hom(a,a)$
        \item For any 1-morphism $f \in Hom(a,b)$ and $g \in Hom(b,c)$, there exists an arrow $g \circ f \in Hom(a,c)$
    \end{enumerate}
    These 1-morphisms must satisfy the same identity and associativity properties as a regular category, now for the 2 morphisms
    \begin{enumerate}
        \item For every 1-morphism $f \in Hom(a,b)$, there exists a 2-morphism $id_f \in Hom2(f,f)$
        \item For every three 1 morphisms $f,g,h \in Hom(a,b)$ if there are  2-morphisms $\alpha \in Hom2(f,g)$ and $\beta \in Hom2(g,h)$, then there exists a 2-morphism $\beta \circ \alpha \in Hom2(f,h)$.
    \end{enumerate}
    And these 2-morphisms must also satisfy the same identity and associativity properties as a regular category. 
\end{definition}
Another way of formalizing a 2-category is a regular category where your HomSets are replaced with HomCategories. Using this we can finally define the category of categories
\begin{definition}
    The 2-Category $Cat$ is the category where
    \begin{itemize}
        \item The objects of $Cat$ are small categories (ie, categories with only a set of morphisms, to avoid problems with Russel's Paradox.)
        \item The 1-morphisms are Functors between categories.
        \item The 2-morphisms are natural transformations between functors.
    \end{itemize}
    
\end{definition}



\subsection{Exercises}
\begin{Exercise}
    Suppose $\alpha: F \Rightarrow G$ is a natural isomorphism. Show that the inverses of the component morphisms define the components of a natural isomorphism $\alpha^{-1}: G \Rightarrow F$.
\end{Exercise}
\begin{Exercise}
    Prove that for any category $\mathcal{A}$ that $\mathcal{A}$ is equivalent to $sk(\mathcal{A})$
\end{Exercise}

\begin{Answer}
    
\end{Answer}
\begin{Exercise}
    Show that for any 2 skeletal categories $A$ and $B$, then an equivalence between $A$ and $B$ forms an isomorphism from $A$ to $B$
\end{Exercise}
\begin{Answer}
    
\end{Answer}

\begin{Exercise}
    Show that for any functor $\mathcal{A} \xrightarrow[]{F} \mathcal{B}$ that the identity natural transformation defined in Definition \ref{def:funccat} is actually a natural transformation that satisfies:
    \[\begin{tikzcd}
	{F(x)} & {F(x)} \\
	{F(y)} & {F(y)}
	\arrow["{F(f)}"', from=1-1, to=2-1]
	\arrow["{\mu_x}", from=1-1, to=1-2]
	\arrow["{\mu_y}"', from=2-1, to=2-2]
	\arrow["{F(f)}", from=1-2, to=2-2]
\end{tikzcd}\]
\end{Exercise}
\begin{Exercise}
    Show that for any functors $\mathcal{A} \xrightarrow[]{F,G,H} \mathcal{B}$, and natural transformations $F \xRightarrow{\alpha} G$ and $G \xRightarrow{\beta}H$ that the composition natural transformation $\beta \circ \alpha$ defined in Definition \ref{def:funccat} is actually a natural transformation that satisfies:
    \[\begin{tikzcd}
	{F(x)} & {F(x)} \\
	{H(y)} & {H(y)}
	\arrow["{F(f)}"', from=1-1, to=2-1]
	\arrow["{\mu_x}", from=1-1, to=1-2]
	\arrow["{\mu_y}"', from=2-1, to=2-2]
	\arrow["{H(f)}", from=1-2, to=2-2]
\end{tikzcd}\]
\end{Exercise}


    


\begin{Exercise}
    Show the category of finite sets $FinSet$ and the category of hereditary finite sets $V_\omega$ are equivalent as categories.
\end{Exercise}
\begin{Answer}
    We can prove this by showing that there are 2 functors $Y$ and $Y'$, and a natural isomorphism between $Y'\circ Y \cong FinSet_{id}$ and $Y \circ Y' \cong (V_\omega)_{id}$ Since every finite set $a$ has a bijection between it and a natural number $|a|$ like so
    \[\begin{tikzcd}
    	a & {|a|}
    	\arrow["{a_*}", curve={height=-6pt}, tail, two heads, from=1-1, to=1-2]
    	\arrow["{a_*^{-1}}", curve={height=-6pt}, tail, two heads, from=1-2, to=1-1]
    \end{tikzcd}\]
    We can then say our functor $Y(a)= |a|$ and for a function $a \rightarrow{f} b$ then $Y(f)$ can be defined as 
    \begin{gather*}
        |a| \xrightarrow{a_*^{-1}} a \xrightarrow{f} b \xrightarrow{b_*} |b|\\
        Y(f) = b_* \circ f \circ a_*^{-1}
    \end{gather*}
    And we can see that it respects composition
    \begin{align*}
        Y(g) \circ Y(f) &= c_* \circ g \circ b_*^{-1} \circ b_* \circ f \circ a_*^{-1}\\
        &= c_* \circ g \circ  f \circ a_*^{-1}\\
        &= Y(g \circ f)
    \end{align*}
    We define $Y'$ using the exact same construction since every hereditary finite set is also a set, we can associate it with a natural number all the same. We also know that if $a$ is in the natural numbers then $a_*$ the identitfy function on $a$ namely $a_{id}$. Notice that since if $|a|$ is a natural number then, $Y(|a|)=|a|$ and if we apply $F$ to a function $|a| \xrightarrow{g}|b|$ then $Y(g)= b_{id} \circ g \circ a_{id}^{-1}=g$, the same goes for $Y'$ applied to any natural number $|a|$ or function between them $g$.

    We can now create the natural isomorphism between $Y'\circ Y \cong FinSet_{id}$. %so we must show a natural isomorphism betwen $Y \cong FinSet_{id}$ where you interpret the outputs of $Y$ inside $FinSet$. 
    Luckily we already have a candidate at the ready, where we can associate every element $a$ with a morphism $a_*$. We now must verify the commutative square saying that 
    \[\begin{tikzcd}
    	{a} & {|a|} \\
    	{b} & {|b|}
    	\arrow["{f}"', from=1-1, to=2-1]
    	\arrow["{a_*}", from=1-1, to=1-2]
    	\arrow["{b_*}"', from=2-1, to=2-2]
    	\arrow["{F'(F(f))}", from=1-2, to=2-2]
    \end{tikzcd}\]
    Alternatively prove that $b_* \circ f = F'(F(f)) \circ  a_*$. Using that $F'(F(f))= b_* \circ f \circ a_*^{-1}$, and we can conclude the proof like so
    \begin{align*}
        F'(F(f)) \circ  a_* &= b_* \circ f \circ a_*^{-1} \circ a_*\\
        &= b_* \circ f 
    \end{align*}
    Thus $Y'\circ Y \cong FinSet_{id}$. Showing that $Y \circ Y' \cong (V_\omega)_{id}$ follows the exact same logic just recasting everything in the domain of hereditary finite sets.
\end{Answer}



