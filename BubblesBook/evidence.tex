\chapter{Proofs and Evidence}

If our goal of mathematics is to try to formalize the validity of proofs, and study proofs as mathematical objects, it is unfortunately somewhat necessary to change our conventional understanding of what it means for a proposition to be true. Namely we need to introduce a primative mathematical notion of "proofs" and "propositions". Where all of the conventional behaviors defined of propositions defined in conventional propositional logic are reworked as rules regarding how properties interact with proofs.

\begin{definition}[Truth]
    A proposition $p$ is true in $\Gamma$ if there exists a proof taking in some assumptions $\Gamma$ and producing a result of $p$.
\end{definition}
However this begs the question, well what are proofs? Well a proof is a primative mathematical object defined to be a list allowed mathematical operations that manipulate the assumptions that are thrown into a proof. Some examples might be helpful

\section{And}

In conventional propositional logic, the conjunction operator "and" is defined by its truth table. However, in our new framework, we need to redefine it in terms of its interaction with proofs.

\begin{definition}[And Introduction]
    If we have both a proof of $p$ and a proof of $q$, then we can construct a proof of $p \land q$.
\end{definition}

This is often denoted as:
\begin{equation*}
    \frac{\pi_p : p, \pi_q : q}{\pi: p \land q}\quad(\text{and-intro})
\end{equation*} \footnote{If one was being pedantic, for every single one of these introduction and elimination rules should assume that the proof of $p$ or $q$, exists using some system of axioms $\Gamma$. So you would write the actual introduction and elimination rules like so $$\frac{ \Gamma \vdash \pi_p : p, \Gamma \vdash \pi_q : q}{ \Gamma \vdash \pi: p \land q}$$}


Where $\pi_p$ and $\pi_q$ are the proofs of $p$ and $q$ respectively. And $\pi$ is the combined proof of $p \land q$

\begin{definition}[And Elimination]
    If we have a proof of $p \land q$, then we can construct a proof of $p$ and a proof of $q$.
\end{definition}

This is often denoted as:


\begin{equation*}
    \frac{\pi : p \land q}{\pi_p : p}\quad(\text{and-left})
\end{equation*}
\begin{equation*}
    \frac{\pi : p \land q}{ \pi_q : q}\quad(\text{and-right})
\end{equation*}


Where $\pi_p$ and $\pi_q$ are the projections of the proof $\pi$ onto $p$ and $q$ respectively.

These rules capture the conventional behavior of the "and" operator, but now in terms of how it interacts with proofs. Note that these rules are not necessarily tied to the truth table of the "and" operator, but rather define how the operator behaves in the context of proofs.

Unfortunately implication is slightly more complicated.
\section{Implication}

The introduction rule for implication states that if we can prove $q$ assuming $p$, then we can construct a proof of $p \Rightarrow q$.

\begin{definition}[Implication Introduction]
    If we have a proof of $q$ from the assumption $p$, then we can construct a proof of $p \Rightarrow q$.
\end{definition}

This is often denoted as:
\begin{equation*}
    \frac{\pi_p : p \vdash \pi_q : q}{\pi: p \Rightarrow q}\quad(\text{implication-intro})
\end{equation*}

Where $\pi_p$ is the proof of $p$, $\pi_q$ is the proof of $q$ from the assumption $p$, and $\pi$ is the constructed proof of $p \Rightarrow q$.

Whats somewhat weird about this is the symbol $\vdash$, although in some ways we have already been using it a bunch, since the following 2 logical expressions

\begin{equation*}
    a : A \vdash b : B \quad \text{is the same as} \quad \frac{a : A}{b : B}
\end{equation*}
mean the same thing. The specifics of entailment and what it means is described in much greater detail in later chapters, so for now we can continue to the elimination rule.

The elimination rule for implication, also known as modus ponens, states that if we have a proof of $p \Rightarrow q$ and a proof of $p$, then we can construct a proof of $q$.

\begin{definition}[Implication Elimination/Modus Ponens]
    If we have a proof of $p \Rightarrow q$ and a proof of $p$, then we can construct a proof of $q$.
\end{definition}

This is often denoted as:
\begin{equation*}
    \frac{\pi : p \Rightarrow q, \pi_p : p}{\pi_q : q}\quad(\text{implication-elim})
\end{equation*}

Where $\pi$ is the proof of $p \Rightarrow q$, $\pi_p$ is the proof of $p$, and $\pi_q$ is the constructed proof of $q$.

There is also a special symbol for propositional equivalence that gets used frequently.
\begin{definition}[If and Only If]
    The proposition $A \iff B$ is short for $(A \implies B) \land (B \implies A)$
\end{definition}
Also oftentimes this symbol $\iff$ will get used a lot as


\section{Or}


The introduction rules for "or" are in some sense mirrored versions of the elimination rules for "and" namely if we have a proof of either $q$, or $p$, then we can construct a proof of $p \lor q$.

\begin{definition}[Or Introduction]
    If we have a proof of $p$, then we can construct a proof of $p \lor q$. Likewise if we have a proof of $q$, then we can construct a proof of $p \lor q$.
\end{definition}

This is often denoted as:
\begin{equation*}
    \frac{\pi_p : p}{\pi : p \lor q}\quad(\text{or-left})
\end{equation*}
\begin{equation*}
    \frac{\pi_q : q}{\pi : p \lor q}\quad(\text{or-right})
\end{equation*}

Where $\pi_p$ and $\pi_q$ is the proof of $p$ and $q$ respectively, and $\pi$ is the constructed proof of $p \lor q$.



The elimination rule for "or" states that if we have a proof of $p \lor q$, and we have proofs of $r$ from the assumptions $p$ and $q$, then we can construct a proof of $r$.

Notice that we get an interesting property from the or introduction rules, since the only way to prove $A \lor B$ is with a proof of $A$ or $B$ we have
\begin{lemma}[Disjunction]
    Specifically if there exists a proof of $A \lor B$ with no assumptions , it implies either there exists a proof of $A$, or there exists a proof of $B$ with no assumptions.
\end{lemma}
At least without introducing further assumptions.

\begin{definition}[Or Elimination]
    If we have a proof of $p \lor q$, and we have proofs of $r$ from the assumptions $p$ and $q$, then we can construct a proof of $r$.
\end{definition}

This is often denoted as:
\begin{equation*}
\frac{\pi : p \lor q, \pi_{p} : p \Rightarrow r, \pi_{q} : q \Rightarrow r}{\pi_r : r}
    \quad(\text{or-elim})
\end{equation*}

Where $\pi$ is the proof of $p \lor q$, $\pi_p$ and $\pi_q$ are the proofs of $p$ implies $r$ and $q$ implies $r$ respectively, letting us conclude with a proof $\pi_r$ of $r$



The elimination rule is a bit more complicated then its "natural dual", the and introduction rule, since we were able to sneakily assume part of the definition of and by using the "," operator in our initial part of and. This is mainly since we need those 2 operators "," and "$\vdash$", to bootstrap everything else. However once we have a more abstract way of formalizing these concepts (ie, category theory) all the beautiful dualities will return!

\section{True and False}
 
 The introduction rule for "True"/$\top$ is straightforward:
 \begin{definition}[True Introduction]
We can always construct a proof of "True".
\end{definition} 
This is often denoted as:

\begin{equation*}
\frac{}{\star : \top}\quad(\text{true-intro})
\end{equation*} 

\begin{lemma}
    The trivially true proposition is true.
\end{lemma}
\begin{proof}
    Use the introduction rule for truth to generate a proof.
\end{proof}

True ($\top$) has no elimination rule. This can be justified by duality by the fact that no information went into our proof of $\top$, so we cant eliminate it to get anything, since otherwise we could generate mathematical information from nothing.

As for False it cannot have an introduction rule  as we cannot construct a proof of "False" without contradicting ourselves. The elimination rule for "False" is more interesting: 
\begin{definition}[False Elimination]
If we have a proof of "False", then we can construct a proof of any proposition $p$.
\end{definition} 
This is often denoted as:
\begin{equation*}
\frac{\pi : \bot}{\pi_p : P}\quad(\text{false-elim})
\end{equation*}
Once can justify this a bit through the lens of conservation of information. Since there is no introduction rule for False (and since any proof of it would create a horrible inconsistency), we can safely use it to prove any proposition we wish. (This is known in philosophy as the principle of explosion.)

Having the False proposition lets us define something interesting
\begin{definition}[Negation]
    The proposition "not p" written as $\neg p$, is defined to be the proposition "$p$ implies False", or $p \implies \bot $. (Assuming some context $\Gamma$)
\end{definition}

This looks really spooky, but it is really just the classic definition of proof by contradiction. An example might be helpful

\begin{example}[$\sqrt{2}$ is irrational]
    \begin{proof}
        Our goal is to prove the statement "$\sqrt{2}$ is not rational", writing this on our system we would have $\neg (\sqrt{2}\text{ is not rational})$. Uing the definition of negation, we want to prove the statement $ (\sqrt{2}\text{ is not rational}) \implies \text{False}$. Using the implication introduction rule, we can see we want to create a proof that assumes that $\sqrt{2}$ is rational and arrives at a false statement. So lettuce begin!
        
        Suppose, that $\sqrt{2}$ is rational. Then, there exist coprime integers $p$ and $q$ such that:
        $$\sqrt{2} = \frac{p}{q}$$
        Squaring both sides, we get:
        $$2 = \frac{p^2}{q^2}$$
        Multiplying both sides by $q^2$, we get:
        $$2q^2 = p^2$$
        This implies that $p^2$ is even, and therefore $p$ is even. Let $p = 2r$ for some integer $r$. Then:
        $$2q^2 = (2r)^2 = 4r^2$$
        Dividing both sides by 2, we get:
        $$q^2 = 2r^2$$
        This implies that $q^2$ is even, and therefore $q$ is even. But this contradicts the assumption that $p$ and $q$ are coprime, giving us our proof of "False" (i.e., a contradiction). 
    \end{proof}
    (This is assuming approximately 20 pages worth of peano arithmetic and number theory, but for now its a good example.)
\end{example}

\begin{definition}[Falsity]
    A proposition $p$ is false in a context $\Gamma$ if their exists a proof taking in our assumptions $\Gamma$ and producing $\neg p$. $(\Gamma \l)$
\end{definition}

\section{Decideability and the Law of Excluded Middle}
\begin{definition}[Decideability]
    A proposition $p$ is decidable if $p \lor \neg p$ is true. (Alternatively saying that $p$ is either true or false.)
\end{definition}
So far our logic is almost identical to the logic taught in a traditional logic class, however it is missing one very important piece.
\begin{axiom}[The Law of Excluded Middle]
    Every proposition $p$ is decidable. (aka for every proposition $p$ then $p \lor \neg p$ is true.)
\end{axiom}
With this axiom the logic we have created on our propositions is identical to that taught in conventional mathematics, with the added benefit of being able to track and verify our proofs!

Its also a very powerful result that you need to prove a bunch of very important results (The intermediate value theorem, an injection from $A \hookrightarrow B$, and an injection from $B \hookrightarrow A$ implies a bijection between $A$ and $B$, every function from the naturals to the naturals is computable, etc...).

Furthermore, even if we dont assume the law of excluded middle we have:

\begin{theorem}[Undecideability must be proven from outside]
    For any proposition $p$, there is no proof showing that $p$ is undecideable. (aka, there is no proof showing $\neg(p \lor \neg p)$)
\end{theorem}
\begin{proof}
    If we want to show that there is no proof of $\neg (p \lor \neg p)$, by the definition of negation, we should assume there does exist a proof of $\neg (p \lor \neg p)$ and strive for a contradiction.

    We can sketch a proof of this real quick, if $p$ was true, then $p$ would be decideable, contradicting our assumption. However, this gives us a proof that $p$ is false, however if $p$ is false, then $p$ is also decideable, contradicting our assumption that it is not.

    That was a lot of words, lets try to break it down using our rules so far, if we know there exists a proof of $\neg (p \lor \neg p)$ and are trying to create a contradiction, it seems like we are at a dead end since we cant prove anything about $p$. What if we start with$\neg p$? 
    By implication introduction we should start with assuming $p$ is true and try to prove something false. If $p$ is true, then we know $p \lor \neg p$ is true by the propetry of or introduction. From here we can use the property of modus ponens and the definition to show that plugging $p \lor \neg p$ into our proof of $\neg( p \lor \neg p)$ gives us $\bot$. Thus we have successfully proved $\neg p$.

    From here we can use our proof of $\neg p$ and or introduction to give us a proof of $p \lor \neg p$. We can again use modus ponens along with our proof of $\neg (p \lor \neg p)$, to create a contradiction. Thus showing our initial assumption that there exists a proof of $\neg (p \lor \neg p)$ was false.
\end{proof}
An advantage of our entire proof framework is that we can take every single one of these steps in this proof and feed it to a computer, so we know our proof has no mistakes

\begin{minted}{text}
theorem not_not_lem (p : Prop) : ¬ ¬ (p ∨ ¬ p) := by
  -- Begin proof by assuming neg_lem : ¬(p or ¬p) and try to show False
  intro neg_lem 
  -- Prove not p by using or introduction along with our assumption.
  have notp : p → False := fun p1 : p => neg_lem (Or.intro_left (¬p) p1)
  -- Prove lem using not p
  have pornotp : p ∨ ¬ p := Or.intro_right (p) notp
  -- Derive a contradiction from neg_lem and lem.
  exact neg_lem pornotp
\end{minted}

This statement (plus some details in the homework) lets us conclude the following result,

\begin{theorem}[Double Negation is Classical Truth]
    Any proposition $\neg \neg P$ is provable from a set of axioms $\Gamma$  $(\Gamma \vdash \neg \neg P)$ if and only if $P$ is provable from $\Gamma$ with the assumption that the law of excluded midddle holds. $(\Gamma, LEM \vdash P)$
\end{theorem}
\begin{proof}
    Use the following solutions to homework problems.  
    \begin{enumerate}
        \item $(\neg \neg P) \land (\neg \neg Q) \iff  \neg \neg (P \land Q)$
        \item $(\neg \neg P) \implies (\neg \neg Q) \iff  \neg \neg (P \implies Q)$
        \item $P \lor \neg P \implies (P \iff \neg \neg P)$
    \end{enumerate}
    Then combine these principles, along with the theorem about the not not law of excluded middle. To go in both directions. (ALSO HOMEWORK)
\end{proof}
\begin{definition}
    A proposition $p$ is called "classically true" if there exists a proof of $\neg \neg p$. (Assuming some context $\Gamma$)
\end{definition}
This lets us know that if we do choose to not assume the law of the excluded middle, we dont actually loose out on any mathematical structure. All that structure is perfectly preserved in the behavior of doubly negated propositions. 

 \begin{Exercise}
Let $P$ be a proposition. Then $P$ implies $\neg\neg P$.
\end{Exercise} \begin{Answer}
We start by assuming $P$ and try to create a proof of $\neg\neg P$. To show that it will suffice to assume $\neg P$ be true and cause a contradiction. Then $\neg P$ is a proof that $P$ is false, but we have assumed $P$ is true. This is a contradiction, so our assumption that $\neg P$ is true must be false. Therefore, $\neg\neg P$ is true.
\end{Answer} \begin{Exercise}
Let $P$ be a proposition. If $P$ is decidable (i.e., either $P$ or $\neg P$ is true), then $P$ is stable (i.e., $P$ is equivalent to $\neg\neg P$).
\end{Exercise} \begin{Answer}
Suppose $P$ is decidable, i.e., either $P$ or $\neg P$ is true. We need to show that $P$ is equivalent to $\neg\neg P$. First, suppose $P$ is true. Then $\neg\neg P$ is true by the previous exercise. Now, suppose $\neg\neg P$ is true. We have two cases:

    Case 1: $P$ is true. Then we are done.
    
    Case 2: $\neg P$ is true. But then this contradicts our assumption $\neg \neg P$. Therefore we can use the false elimination rule and conclude that $P$ must be true.

In both cases, we have shown that $P$ is equivalent to $\neg\neg P$, so $P$ is stable.
\end{Answer} \begin{Exercise}
Let $P$ and $Q$ be propositions. If $P$ implies $Q$, then $\neg Q$ implies $\neg P$.
\end{Exercise} \begin{Answer}
Suppose $P$ implies $Q$. Let $\neg Q$ be true. We need to show that $\neg P$ is true. Suppose, for the sake of contradiction, that $P$ is true. Then, by our assumption, $Q$ is true. But this contradicts our assumption that $\neg Q$ is true. Therefore, our assumption that $P$ is true must be false, and we conclude that $\neg P$ is true.
\end{Answer} \begin{Exercise}
Let $P$ be a proposition. Then $\neg P$ is equivalent to $\neg\neg\neg P$.
\end{Exercise} \begin{Answer}
We need to show that $\neg P$ implies $\neg\neg\neg P$ and vice versa. 

\paragraph{Forward Direction $\neg P \implies \neg\neg\neg P$} 


\paragraph{Backward Direction $\neg\neg\neg P \implies \neg P$}


\end{Answer}

\begin{Exercise}
For any two propositions $P$ and $Q$, the following equivalence holds:
$$(\neg P) \land (\neg Q) \iff \neg (P \lor Q)$$
\end{Exercise}
\begin{Answer}
\begin{proof}
We split this proof into 2 sections
\paragraph*{Forward Implication}
Suppose we have $(\neg P) \land (\neg Q)$. We need to show that $\neg (P \lor Q)$.

Let's assume, for the sake of contradiction, that $P \lor Q$ holds. Then, by definition of "or", we have two cases:

\begin{enumerate}
\item $P$ holds. But then we have a contradiction, since we have $\neg P$ as the left part of our assumption  $(\neg P) \land (\neg Q)$.
\item $Q$ holds. But then we have a contradiction, since we have $\neg Q$ as the right part of our assumption  $(\neg P) \land (\neg Q)$.
\end{enumerate}

In both cases, we reach a contradiction, so our assumption that $P \lor Q$ holds must be false. Therefore, we have shown that $\neg (P \lor Q)$.

\paragraph*{Backward Implication}
Suppose we have $\neg (P \lor Q)$. We need to show that $(\neg P) \land (\neg Q)$.

Let's assume $P$ holds. Then, by definition of "or", we have $P \lor Q$, which contradicts our assumption that $\neg (P \lor Q)$. Therefore, $P$ cannot hold, so we must have $\neg P$.

Similarly, let's assume $Q$ holds. Then, by definition of "or", we have $P \lor Q$, which again contradicts our assumption that $\neg (P \lor Q)$. Therefore, $Q$ cannot hold, so we must have $\neg Q$.

Since we have shown both $\neg P$ and $\neg Q$, we have established that $(\neg P) \land (\neg Q)$.

This completes the proof of the equivalence.
\end{proof}
\end{Answer}

\begin{Exercise}
Prove that $(\neg P) \vee (\neg Q) \Rightarrow \neg (P \wedge Q)$.
\end{Exercise} 
\begin{Answer}

Since we are trying to prove a contradiction, let us assume we have a proof $h$ that either $P$ is false or $Q$ is false (i.e., $(\neg P) \vee (\neg Q)$), and try to derive a proof of $\neg( P \wedge Q)$. To prove this we first assume we have a proof $pq$ that both $P$ and $Q$ are true (i.e., $P \wedge Q$) and try to arrive at a contradiction. We can do this by doing a proof by cases on $h$.

\begin{itemize}
\item If $P$ is false (i.e., $\neg P$), then we can obtain a proof $p$ that $P$ is true from $pq$. But this contradicts our assumption that $P$ is false, so we can conclude that our initial assumption (that both $P$ and $Q$ are true) is false.
\item If $Q$ is false (i.e., $\neg Q$), then we can obtain a proof $q$ that $Q$ is true from $pq$. But this contradicts our assumption that $Q$ is false, so we can conclude that our initial assumption (that both $P$ and $Q$ are true) is false.
\end{itemize} 

In either case, we have shown that if $(\neg P) \vee (\neg Q)$, then $\neg (P \wedge Q)$.
\end{Answer}
