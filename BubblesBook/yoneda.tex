\chapter{Yoneda's Lemma and Presheaves}

Yoneda's lemma often takes 2 forms, the first form begins like so

\begin{definition}[Hom Functor V1]
    For a category $\mathcal{C}$ every object in $c \in C$ gives rise to a functor $Hom(c,\_)$ from $C \xrightarrow[]{Hom(c,\_)} Set$. 
    Such that for every $x \in C$ maps $x$ to the hom set $Hom(c,x)$ and for every function $x \xrightarrow[]{f} y$ induces a map $Hom(c,x) \xrightarrow{f \circ \_} Hom(c,y)$ that takes an arrow and composes it with $f$ like so:
    % https://q.uiver.app/#q=WzAsMyxbMCwwLCJjIl0sWzEsMCwieCJdLFsyLDAsInkiXSxbMSwyLCJmIl0sWzAsMSwiXFxfIl0sWzAsMiwiZiBcXGNpcmMgXFxfIiwyLHsiY3VydmUiOjF9XV0=
\[\begin{tikzcd}
	c & x & y
	\arrow["f", from=1-2, to=1-3]
	\arrow["{\_}", from=1-1, to=1-2]
	\arrow["{f \circ \_}"', curve={height=16pt}, from=1-1, to=1-3]
\end{tikzcd}\]
Giving us the following functor:
% https://q.uiver.app/#q=WzAsNixbMCwwLCJ4Il0sWzAsMSwieSJdLFsxLDAsIlxcbWFwc3RvIl0sWzEsMSwiXFxtYXBzdG8iXSxbMiwwLCJIb20oYyx4KSJdLFsyLDEsIkhvbShjLHkpIl0sWzAsMSwiZiJdLFs0LDUsImYgXFxjaXJjIFxcXyJdXQ==
\[\begin{tikzcd}
	x & \mapsto & {Hom(c,x)} \\
	y & \mapsto & {Hom(c,y)}
	\arrow["f", from=1-1, to=2-1]
	\arrow["{f \circ \_}", from=1-3, to=2-3]
\end{tikzcd}\]
Dually for every object $c$ we also have the functor $Hom(\_,c)$ giving us a functor from $\mathcal{C}^{op} \xrightarrow[]{Hom(_,c)} Set$
\end{definition}

\begin{theorem}[Yoneda's Lemma]
    Pick a category $\mathcal{C}$ and a functor $F$ between $\mathcal{C} \xrightarrow[]{F} Set$. Using the definition above an element $c \in \mathcal{C}$ also defines a functor $\mathcal{C} \xrightarrow[]{Hom(_,c)} Set$.  Notice that these functors are now parallel and we can define natural transformations from $Hom(c,\_))$ to $F$
    % https://q.uiver.app/#q=WzAsMixbMCwwLCJcXG1hdGhjYWx7Q30iXSxbMiwwLCJTZXQiXSxbMCwxLCJIb20oYyxcXF8pIiwwLHsiY3VydmUiOi0yfV0sWzAsMSwiRiIsMix7ImN1cnZlIjoyfV0sWzIsMywiIiwwLHsic2hvcnRlbiI6eyJzb3VyY2UiOjIwLCJ0YXJnZXQiOjIwfX1dXQ==
\[\begin{tikzcd}
	{\mathcal{C}} && Set
	\arrow[""{name=0, anchor=center, inner sep=0}, "{Hom(c,\_)}", curve={height=-12pt}, from=1-1, to=1-3]
	\arrow[""{name=1, anchor=center, inner sep=0}, "F"', curve={height=12pt}, from=1-1, to=1-3]
	\arrow[shorten <=3pt, shorten >=3pt, Rightarrow, from=0, to=1]
\end{tikzcd}\]
Yoneda's lemma states that the cardinally of the set of all natural transformations between $Hom(c,\_)$ and $F$ is the same cardinality as the set $F(c)$.

Furthermore there is a well defined isomorphism between the 2 sets
\begin{align*}
    Hom(Hom(c,\_),F) \cong F(c)
\end{align*}
Defined by $ Hom(Hom(c,\_),F) \xrightarrow[]{\Psi} F(c)$. Remember that for any natural $\alpha$ between $Hom(Hom)$ 

\end{theorem}
\begin{definition}[Hom Functor V2]
    Another way to think about the Hom functor is as a map between the category $\mathcal{C}^{op} \times  \mathcal{C} \xrightarrow{Hom} Set$. (Remember product categories consist of pairs of objects $(a,b)$ one from each category, and have arrows that are pairs of arrows between objects $(a,b) \xrightarrow[]{(f,g)} (c,d)$).  

    In particular $Hom$ takes the pair $(a^{op},b)$ to the set $Hom(a,b)$. And for every arrow $(a^{op},b) \xrightarrow[]{(f^{op},g)} (c^{op},d)$, should induce an arrow from $Hom(a,b)$ to $Hom(c,d)$.  Using the fact that $a^{op} \xrightarrow[]{f^{op}} c^{op}$ is just an arrow $c \xrightarrow[]{f} a$. Given an abitrary morphism from $a$ to $b$ we can create a morphism from $c$ to $d$ by composing $f$ and $g$ on both sides like so:
    % https://q.uiver.app/#q=WzAsNCxbMCwwLCJhIl0sWzEsMCwiYiJdLFswLDEsImMiXSxbMSwxLCJkIl0sWzAsMSwiXFxfIl0sWzIsMywiZyBcXGNpcmMgXFxfIFxcY2lyYyBmIiwyXSxbMiwwLCJmIl0sWzEsMywiZyJdXQ==
\[\begin{tikzcd}
	a & b \\
	c & d
	\arrow["{\_}", from=1-1, to=1-2]
	\arrow["{g \circ \_ \circ f}"', from=2-1, to=2-2]
	\arrow["f", from=2-1, to=1-1]
	\arrow["g", from=1-2, to=2-2]
\end{tikzcd}\]
    Note that using the fact that product categories and functor categories form exponentials in $Cat$ we can say the functor $\mathcal{C}^{op} \times  \mathcal{C} \xrightarrow{Hom} Set$ gives rise to two other functors using the exponential property:
    \begin{gather*}
        \mathcal{C} \xrightarrow{Yo} Set^{\mathcal{C}^{op}}\\
        \mathcal{C}^{op} \xrightarrow{Yo^{op}} Set^{\mathcal{C}}
    \end{gather*}
    The first functor is called the "Yoneda Embedding"
\end{definition}

\begin{definition}
    A functor $A \xrightarrow{F} B$ defines a map between $Hom(a_1,a_2) \xrightarrow[]{f^\* } Hom(f(a_1),f(a_2))$ for all pairs $a_1,a_2 \in A$. Then $F$ is:
    \begin{itemize}
        \item Faithful if for all $Hom(a_1,a_2) \xrightarrow[]{f^*} Hom(f(a_1),f(a_2))$, $f^*$ is surjective.
        \item Full if for all $Hom(a_1,a_2) \xrightarrow[]{f^*} Hom(f(a_1),f(a_2))$, $f^*$ is injective.
        \item Full and faithful if for all $Hom(a_1,a_2) \xrightarrow[]{f^*} Hom(f(a_1),f(a_2))$, $f^*$ is a bijection.
    \end{itemize}
\end{definition}
Using all this we can write a different often more convinent form of the yoneda lemma:
\begin{theorem}[Yoneda's Lemma]
    The functors 
   \begin{gather*}
        \mathcal{C} \xrightarrow{Yo} Set^{\mathcal{C}^{op}}\\
        \mathcal{C}^{op} \xrightarrow{Yo^{op}} Set^{\mathcal{C}}
    \end{gather*}
    are full and faithful.
\end{theorem}
These enable us to further talk about what exactly these things are
\begin{definition}
    A presheaf of a category $C$, is the category $Set^{C^{op}}$
\end{definition}
\begin{definition}
    A presheaf of a category $C$ ($Set^{C^{op}}$) is the "free cocompletion" of the category $C$. The "free cocompletion" can be conceptualized as "freely adjoining colimits to C" What this means is that given:
    \begin{enumerate}
        \item For any category $D$ with all colimits.
        \item And a functor $C \xrightarrow{F} D$
    \end{enumerate}
    Then there is a unique functor $\bar{F}$ from $Set^{C^{op}} \xrightarrow[]{\bar{F}} D$ that:
    \begin{enumerate}
        \item Preserves colimits.
        \item Preserves the following commutative diagram:
    \end{enumerate}
    % https://q.uiver.app/#q=WzAsMyxbMCwwLCJDIl0sWzEsMCwiU2V0XntDXntvcH19Il0sWzIsMCwiRCJdLFswLDEsIllvIl0sWzEsMiwiXFxiYXJ7Rn0iXSxbMCwyLCJGIiwyLHsiY3VydmUiOjJ9XV0=
\[\begin{tikzcd}
	C & {Set^{C^{op}}} & D
	\arrow["Yo", from=1-1, to=1-2]
	\arrow["{\bar{F}}", from=1-2, to=1-3]
	\arrow["F"', curve={height=12pt}, from=1-1, to=1-3]
\end{tikzcd}\]
\end{definition}
Also crucially for our understanding of topos theory.
\begin{theorem}[Presheaves are Sheaves]
    For a small $C$, $Set^C$ is a topos.
\end{theorem}




\subsection*{Exercises}

\begin{Exercise}
    Prove that given a Full Functor $A \xrightarrow[]{F} B$. Then for any 2 elements $x,y \in A$ if $F(x) \cong F(y)$, then $x \cong y$
\end{Exercise}



\begin{Exercise}
    Prove Yoneda Consequence \#1, namely if for all $x \in \mathcal{C}$ if $Hom(x,a) \cong Hom(x,b)$ then $a \cong b$. (Use the definition of a yoneda embedding)
\end{Exercise}