
\begin{definition}[Well Order]
    A well order on a set $X$ is a partial order on $X$ where every subset of $X$ has a least element.
\end{definition}
Here are some fun facts about ordinals
\begin{enumerate}
    \item The class of all ordinals is well ordered under set inclusion $(\in)$
    \item An ordinal is the set of all ordinals strictly less than it. (The first ordinal is the empty set.)
    \item Any set with a well order is order isomorphic to exactly 1 ordinal.
\end{enumerate}

\begin{theorem}
    For any set $\alpha$, consider the ordinal $\omega_\alpha$ to be the smallest ordinal with cardinality $|\alpha|$. Then $\omega_\alpha$ is order isomorphic to $\omega_\alpha \times \omega_\alpha$ under the following order
    $$
(x, y)<(w, z) \Longleftrightarrow\left\{\begin{array}{l}
\max \{x, y\}<\max \{w, z\} \\
\max \{x, y\}=\max \{w, z\} \wedge x<w \\
\max \{x, y\}=\max \{w, z\} \wedge x=w \wedge y<z
\end{array}\right.
$$
\end{theorem}
\begin{proof}
    Firstly, it is obvious that $k$ is at most of the order type of $K \times K$ since the order type of $k$ can be simply be written as $\alpha \mapsto(\alpha, \alpha)$. The other direction we prove by induction on $\alpha$ that for the initial ordinal $\omega_\alpha$ it is true: $\omega_\alpha=\omega_\alpha \times \omega_\alpha$.

Fact: If $\delta<\omega_\alpha$ (where $\omega_\alpha$ is the $\alpha$-th initial ordinal) then $|\delta|<N_\alpha$.

The claim is true for $\omega_0=\omega$ since for any $k$ the set $\{(n, m) \mid(n, m)<(k, k)\}$ is finite. Therefore the order type of $\omega \times \omega$ is the supremum of $\left\{k_n \mid n \in \omega\right\}$ and $k_n$ are finite. Simply put, the order type is $\omega$.

Now assume (by contradiction) $\alpha$ was the least ordinal such that $\omega_\alpha$ was a counterexample to this claim, i.e. $\omega_\alpha$ is strictly less than the order type of $\omega_\alpha \times \omega_\alpha$.

Let $(\gamma, \beta)<\omega_\alpha \times \omega_\alpha$ be the pair of ordinals such that the order type of $\{(\xi, \zeta) \mid(\xi, \zeta)<(\gamma, \beta)\}$ is $\omega_\alpha$.
Take $\delta$ such that $\omega_\alpha>\delta>\max \{\gamma, \beta\}$ then $(\gamma, \beta)<(\delta, \delta)$ and in particular $\{(\xi, \zeta) \mid(\xi, \zeta)<(\delta, \delta)\}$ has cardinality of at least $\omega_\alpha$, as it extends a well order of the type $\omega_\alpha$.

However, $\delta<\omega_\alpha$ by the fact above it is of smaller cardinality, and thus that set has the cardinality $|\delta| \times|\delta|=|\delta|<\omega_\alpha$ by our induction assumption. Hence, a contradiction.
\end{proof}



Let us overview the theorem of Zermelo, namely if the axiom of choice holds then $K=K^2$ for every infinite K.

This is fairly simple, by the canonical well ordering of pairs.
Consider $\alpha \times \beta$, this can be well ordered as ordinal multiplication (that is $\beta$ copies of $\alpha$, i.e. lexicographical ordering), or it can be ordered as following:


This is a well-ordering (can you see why?). Now we will prove that $K \times K$ has the same order type as $K$, this is a proof that the two sets have the same cardinality, since similar order types induce a bijection.


%\newpage