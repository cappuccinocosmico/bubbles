\begin{definition}
    For any categories $A$, $B$, and an element $c \in B$, there is a constant functor $A \xrightarrow[]{\delta c} B$ that maps every object in $A$ onto $c$ and every morphism in $A$ onto $id_c$
\end{definition}

Consider a map from small category, that looks like a diagram
% https://q.uiver.app/#q=WzAsOSxbMCwwLCJcXGJ1bGxldCJdLFswLDEsIlxcYnVsbGV0Il0sWzAsMiwiXFxidWxsZXQiXSxbMSwwLCJcXG1hcHN0byJdLFsxLDEsIlxcbWFwc3RvIl0sWzEsMiwiXFxtYXBzdG8iXSxbMiwwLCJBIl0sWzIsMSwiQiJdLFsyLDIsIkMiXSxbMCwxXSxbMSwyXSxbNiw3LCJmIiwyXSxbNyw4LCJnIiwyXV0=
\[\begin{tikzcd}
	\bullet & \mapsto & x \\
	\bullet & \mapsto & y \\
	\bullet & \mapsto & z
	\arrow[from=1-1, to=2-1]
	\arrow[from=2-1, to=3-1]
	\arrow["f"', from=1-3, to=2-3]
	\arrow["g"', from=2-3, to=3-3]
\end{tikzcd}\]

And now consider the constant functor for some $c$. Then what would a natural transformation $\mu$ from $F$ to $\delta c$ would be a 



\begin{comment}
\begin{Exercise}
    \begin{itemize}
        \item [(a)] Let $A$ and $B$ be finite sets. Show that their union $A \cup B$ is finite.
        
        \item [(b)] Show that the union of finitely many finite sets is finite.
    \end{itemize}
    
\end{Exercise}
\begin{Answer}
    We shall use category theory for this proof in an attempt to prove that it is a practical branch of mathematics that can be used to prove real results.\footnote{Also because being a contrarian is fun :)} Since we have a grothendiek universe $V_\omega$ we know that any union of two hereditary finite sets is hereditary finite, but we dont know the same about finite sets in general. 
    
    Our method for proving this will be to use some category theory, to first describe how the union of 2 sets, $A\cup B$ is a categorical construction, then since categorical constructions respect equivalences if the category of finite sets $FinSet$ is equivalent to $V_\omega$ then since $V_\omega$ is closed under unions then so is $FinSet$.
    
    So how would we construct the set $A \cup B$ using only the properties of functions?  $A$ and $B$ their union can be determined without using set inclusion by using the fact that $A \cap B$ is a subset of $A$ and $B$ and therefore has an inclusion function $A\cap B \stackrel{f}{\hookrightarrow} A$ and $A\cap B \stackrel{f}{\hookrightarrow} B$. We can then construct the union $A\cup B$ by saying that its the "smallest" set where there the inclusion function $A \xrightarrow{f^*} A \cup B$ and $B \xrightarrow{g^*} A \cup B$, such that $g^* \circ g = f^* \circ f$.
   \[\begin{tikzcd}
    	A & {A \cup B} \\
    	{A\cap B} & B
    	\arrow["f", tail, from=2-1, to=1-1]
    	\arrow["g"', tail, from=2-1, to=2-2]
    	\arrow[dashed, tail, from=1-1, to=1-2]
    	\arrow[dashed, tail, from=2-2, to=1-2]
    	\arrow["\lrcorner"{anchor=center, pos=0.125, rotate=90}, draw=none, from=2-1, to=1-2]
    \end{tikzcd}\]
     This can be formalized using the concept of co-limits and pullbacks, but what exactly this construction is doesnt end up mattering since all it needs to show us is that you can construct construct $A\cup B$ without any inclusion or subset rules if you define $A$,$B$ and $A\cup B$ and the appropriate inclusion functions. 

    All that needs to be shown is that $A\cap B$ is finite, luckily since $A \cap B$ is a subset of a finite set $A$ then it is finite as shown in the previous exercise.

    A finite union can also be represented as a finite categorical construction, but it might be simpler to bail out and use induction. Consider $A$ to be the subset of integers where a union of $n$ finite sets is finite. We know that $0$ in a since a union of no sets is the empty set. If $n \in A$ then consider a family of sets $\{X_i | i \in n^+\}$, then $\bigcup_{i \in n} X_i$ is finite and since the union of 2 sets is finite we know that
    \begin{align*}
        \bigcup_{i \in n^+} X_i &= X_{n^+}\bigcup_{i \in n^+} X_i
    \end{align*}
    is finite. Thus by induction $A=\mathbb{N}$ and we are done.
\end{Answer}

\end{comment}

