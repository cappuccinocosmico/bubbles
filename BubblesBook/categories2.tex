\chapter{Categories II:Skeletons, Functors and Equivalences}
\begin{definition*}[Skeletal Category]
    A Category $C$ is skeletal if 2 objects $a,b$ are isomorphic then $a =b$, aka $a \cong b \iff a =b$
\end{definition*}
\begin{definition*}[Skeleton Category w/ Axiom of Choice]
    A Skeleton of a Category $\mathcal{C}$ ($sk(\mathcal{C})$) is a skeletal category constructed like so:
    \begin{enumerate}
        \item Taking all objects in $\mathcal{C}$ and defining an equivalence relation on objects defined by 2 isomorphisms on those objects.
        \item Create a set $Obj(sk(\mathcal{C}))$ of representatives from each equivalence/isomorphism class.
        \item Construct the category $sk(\mathcal{C}$, where the objects are the representatives from the equivalence classes, and all the arrows between members of $Obj(sk(\mathcal{C}))$ are inherited from $\mathcal{C}$
    \end{enumerate}
\end{definition*}
\begin{example}
A skeleton of $FinSet$ is the category $\mathbb{N}$, where the objects are the natural numbers constructed with peano arithmetic $(\phi, \{\phi \}, \{\phi,\{\phi\}\})$, and the arrows are standard set functions.
\end{example}
\begin{definition*}
    A Functor $F$ between two categories $A$ and $B$ maps every object $x \in \mathcal{O}(A)$ to an object $F(x) \in \mathcal{O}(B)$, and maps every arrow $x \xrightarrow{f} y$ to an arrow $F(x) \xrightarrow{F(f)} F(y)$ in $B$ in a way that preserves the domain and codomain. Such that 
    \begin{itemize}
        \item $F(g \circ f)= F(g) \circ F(y)$
        \item $F(id_a)=id_{F(a)}$
    \end{itemize}
    (You can think of them as the homomorphisms of categories) 
\end{definition*}
\begin{definition*}
    For every category $A$ there is a functor $id_A: A \Rightarrow A$ that for every object $x \in A$ then $id_A(x)=x$ and every function $x \xrightarrow[]{f} y$ then $id_A(f)=f$
    
    Categories $A$ and $B$ are isomorphic if there is a functor $F: A \Rightarrow B$ and an inverse functor $F^{-1}: B \Rightarrow A$ such that $F \circ F^{-1}=id_B$ and $F^{-1}\circ F = id_A$
\end{definition*}

\begin{adefinition}[Equivalence of Categories]
    2 Categories $A$ and $B$ are equivalent if and only if $sk(A)$ is isomorphic to $sk(B)$
\end{adefinition}
\begin{theorem}[Fundamental Theorem of Cat Theory 1]
    Any construction on categories is going to respect the equivalence of categories.
\end{theorem}

\section{Exercises}
\begin{Exercise}
    Show that functors preserve isomorphisms.

    \begin{proof}
    Suppose that between categories $A$ and $B$ there exists a functor $F: A \rightarrow B$ suppose that for $x,y \in A$ there exists an isomorphism $f: x \rightarrow y$ with inverse $g: y \rightarrow x$. Then for arrows $F(f): F(x) \rightarrow F(y)$ and $F(g): F(y) \rightarrow F(x)$, it is the case that

    \begin{align*}
       F(f)\circ F(g) &= F(f\circ g) \quad \text{ def functor} \\ &= F(id_y) \quad \text{ def isomorphism } \\
        &= id_{F(y)} \quad \text{ def functor}.
    \end{align*}

    Likewise, 

    \begin{align*}
       F(g)\circ F(f) &= F(g\circ f) \quad \text{ def functor} \\ &= F(id_x) \quad \text{ def isomorphism } \\
        &= id_{F(x)} \quad \text{ def functor} 
    \end{align*}
    \end{proof}

    Since $F(f)\circ F(g) = id_{F(y)} $ and $F(g)\circ F(f) = id_{F(x)}$ it follows by def. isomorphism that $F(x) \cong F(y)$. 
\end{Exercise}


\begin{Answer}
    Consider a functor $F: \mathrm{C} \rightarrow \mathrm{D}$ and an isomorphism $f: x \rightarrow y$ in $\mathrm{C}$ with inverse $g: y \rightarrow x$. Applying the two functoriality axioms:
$$
F(g) \circ F(f)=F(g \circ f)=F\left(id_x\right)=id_{F x} .
$$
Thus, $ F (y) \xrightarrow{F (g)} F x$ is a left inverse to $ F x \xrightarrow{F (g)} F y$. Exchanging the roles of $f$ and $g$ (or arguing by duality) shows that $F g$ is also a right inverse.
\end{Answer}




 
\begin{Exercise}
Try to construct a functor from Set (The category of sets) into Grp (The category of Groups) that will map a set to some group, where the elements of the set will correspond to the generators of the group. (This will involve picking a group that will allow its generators to be arbitrarily shuffled around). Hint\footnote{The biggest issue is coming up with a group without any structure aside from its generators, consider a group with generators $\{a,b,c,d\}$ where any string of $\{a,a^{-1},b,b^{-1},c,c^{-1},d,d^{-1}\}$, like $abcd^{-1}adb^{-1}$ is a valid group element, where composition is given by concatenation and then canceling inverses that are next to each other.}
\end{Exercise}
\begin{Exercise}
    Consider a group $z$ inside the category of groups (or more generally an object $z$ inside any category $\mathcal{C}$). Let us define a functor $Hom(z,\_): \mathcal{C} \rightarrow Set$, that maps an object $x \in \mathcal{C}$ to the set of homomorphisms/arrows from $z$ to $x$.  And for every homomorphism/arrow $x \xrightarrow[]{f} y$. Corresponds to a function taking every homomorphism/arrow from $z \xrightarrow[]{g} x$ and mapping it to the arrow  $z \xrightarrow[]{f \circ g} y$ like so:
    % https://q.uiver.app/#q=WzAsMyxbMCwwLCJ6Il0sWzAsMSwieCJdLFsxLDEsInkiXSxbMCwxLCJnIiwyXSxbMCwyLCJmIFxcY2lyYyBnIl0sWzEsMiwiZiIsMl1d
\[\begin{tikzcd}
	z \\
	x & y
	\arrow["g"', from=1-1, to=2-1]
	\arrow["{f \circ g}", from=1-1, to=2-2]
	\arrow["f"', from=2-1, to=2-2]
\end{tikzcd}\]
    Show that this functor satisfies the properties of functors namely $Hom(z,\_)(id_x)= id_{Hom(z,\_)(x)}$ and $Hom(z,\_)(f \circ g) =Hom(z,\_)(f)  \circ Hom(z,\_)(g)  $ 

    \begin{proof}
        
    \end{proof}
\end{Exercise}







